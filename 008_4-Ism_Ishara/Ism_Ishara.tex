% Generated by LaTreeX, <https:\\lautgesetz.com/latreex/>
% !TEX root = ../000_Complet/A4_008_4.tex
\documentclass{article}
\usepackage[a4paper,margin=1in,landscape]{geometry}
\usepackage{ulem}
\usepackage[active,tightpage,xetex]{preview}
\usepackage[tuenc]{fontspec}
\usepackage{pstricks,pst-node,pst-tree}
\usepackage{varwidth,realscripts}
\usepackage{graphicx}
\usepackage{bidi}
\usepackage{titling}
\defaultfontfeatures{Mapping=tex-text}
\setmainfont{Amiri-Bold}[Script=Arabic,Scale=1.1]
\pagestyle{empty}
\psset{showbbox=false,
       treemode=D,
       linewidth=0.3pt,
       treesep=2em,
       levelsep=3em,
       arrows=->,
       edge=\ncangles,
       angleA=-90,
       angleB=90,
       armA=1.5em,
       armB=1.5em}
\newcommand{\LFTw}[2]{%
\Tr[name=#1,ref=t]{\psframebox[linestyle=none,framesep=3pt]{%
  \begin{varwidth}{10em}\center #2\end{varwidth}}}
}
\newcommand{\LFTwl}[2]{%
\Tr[ref=#1]{\psframebox[linestyle=none,framesep=3pt]{%
  \begin{varwidth}{15em}\center #2\end{varwidth}}}
}
\newcommand{\LFTwsolid}[2]{%
\Tr[ref=#1]{\psframebox[linestyle=solid,framesep=3pt]{%
  \begin{varwidth}{30ex}\center #2\end{varwidth}}}
}
\newcommand{\LFTr}[2]{%
\Tr[ref=#1]{%
  \psframebox[linestyle=none,framesep=2pt]{#2}}
}

\newfontfamily{\exemplefont}{Amiri}[Scale=1,Script=Arabic]
\newfontfamily{\moyennefont}{Amiri-Bold}[Scale=1,Script=Arabic]
\newcommand{\exemple}[1]{%
  \fontspec{Amiri}[Scale=1,Script=Arabic]
  \RL{#1}
}
\newcommand{\annot}[1]{%
  \fontspec{Amiri}[Scale=0.9,Script=Arabic]
  \RL{#1}
}

\newcommand{\emphaseVV}[1]{\char"200D\textcolor{black!40}{\char"200D #1\char"200D}{\char"200D}}
\newcommand{\emphaseVF}[1]{\char"200D\textcolor{black!40}{\char"200D #1}}
\newcommand{\emphaseFV}[1]{\textcolor{black!40}{#1\char"200D}{\char"200D}}
\newcommand{\emphaseFF}{\textcolor{black!40}}

\def\pstreehooki{\psset{thislevelsep=*0pt}}
\def\pstreehookiii{\psset{thislevelsep=*0pt}}
\def\pstreehookv{\psset{thislevelsep=*0pt}}
\def\pstreehookvii{\psset{thislevelsep=*0pt}}
\def\pstreehookix{\psset{thislevelsep=*0pt}}
\def\pstreehookxi{\psset{thislevelsep=*0pt}}
\def\pstreehookxiii{\psset{thislevelsep=*0pt}}
\def\pstreehookxv{\psset{thislevelsep=*0pt}}

\pretitle{}\posttitle{}\title{\RL{اِسم مَعرِفَة}}
\preauthor{}\postauthor{}
\author{}
\predate{}\postdate{}\date{}

\begin{document}
\begin{preview}
  \begin{center}
    
    \normalsize %small,normalsize,large,Large,LARGE,huge,HUGE
    %pstree
    % \\ \begin{flushright}{\exemplefont  } \end{flushright}  \emphaseVV{}
    \pstree{\LFTw{t}{\RL{اِسمُ الإِشَارَة}}}{\pstree{\Tp[edge=none]}{%

        \pstree{\LFTw{t}{\RL{مَكَانُ المُشَارِ إِلَيهِ}}}{\pstree{\Tp[edge=none]}{%
            \pstree{\LFTwl{t}{\RL{بَعِيد}\\\moyennefont \RL{وجِبَ اقتِرَانُهُ بكَاف الخِطَاب وجوباً}}}{\pstree{\Tp[edge=none]}{%
                \LFTw{t}{\RL{مُقتَرِن بَاللَّام}\\\exemplefont \RL{ذَلِكَ}}
                \LFTw{t}{\RL{مُجَرَّد مِنَ اللَّام}\\\exemplefont \RL{ذَاكَ}}
              }}
            \pstree{\LFTw{t}{\RL{قَرِيب}}}{\pstree{\Tp[edge=none]}{%
                \LFTw{t}{{\moyennefont \RL{إِذَا لَم يَتَّصِل بِهِ حَرفُ التَّنبِيه}}\\\RL{يُلحَق بِهِ الكَافُ جَوَازاً}\\\exemplefont \RL{جَاءَنِي ذَا}\\\RL{جَاءَنِي ذَاكَ}}
                \LFTw{t}{{\moyennefont \RL{إِذَا اِتَّصَلَ بِهِ حَرفُ التَّنبِيه}\\\RL{(الهَاء)}}\\\RL{مُجَرَّد مِنَ الكَاف وجوباً}\\\exemplefont \RL{جَاءَنِي هَذَا}\\\sout{\RL{جَاءَنِي هَذَاكَ}}}
              }}
          }}
        \pstree{\LFTw{t}{\RL{حَسَبَ المُشَارِ إِلَيهِ}}}{\pstree{\Tp[edge=none]}{%
            \LFTwl{t}{\RL{جَمع}\\\begin{flushright} \RL{أُولى (الجَمع فِي لُغَةِ القَصر) }\\ \RL{أولاَءِ}\\ \RL{أُولَئِكَ}\\ \RL{هَؤلاَءِ}\end{flushright}}
            \pstree{\LFTw{t}{\RL{مُثَنَّى}}}{\pstree{\Tp[edge=none]}{%
                \LFTw{t}{\RL{تَان}\\ \RL{تَين}}\tbput[labelsep=-5pt]{\annot {مؤَنَّث}}
                \LFTw{t}{\RL{ذَان}\\ \RL{ذَين}}\tbput[labelsep=-5pt]{\annot {مُذكَّر}}
              }}
            \pstree{\LFTw{t}{\RL{مُفرَد}}}{\pstree{\Tp[edge=none]}{%
                \LFTw{t}{
                  \begin{tabular}{l|l}
                    ذَات & ذِي  \\
                    تِي  & ذِهِي \\
                    تِهِي & ذِهِ  \\
                    تِهِ  & ذِهْ  \\
                    تِهْ  &
                  \end{tabular}}\tbput[labelsep=-5pt]{\annot {مؤَنَّث}}
                \LFTw{t}{ \RL{ذَا}}\tbput[labelsep=-5pt]{\annot {مُذكَّر}}
              }}
          }}
      }}
    \begin{center}\line(1,0){450}\end{center}
    \pstree{\LFTw{t}{\RL{اِمتِنَاع اللَّام}}}{\pstree{\Tp[edge=none]}{%
        \LFTw{t}{\RL{إِذَا تَقَدَّمَت هَاءُ التَّنبِيه}\\\exemplefont \RL{هَذَاكَ، ذَالِكَ}\\\sout{\RL{هَذَالِكَ}}}
        \LFTw{t}{\RL{الجَمع فِي لُغَةِ مَنْ مَدَّه}\\\exemplefont \RL{ أُولَالِكَ }\\\sout{\RL{أُولاَءِلَكَ}}}
        \LFTw{t}{\RL{المُثَنَّى}\\\exemplefont \RL{ذَانِك، تَانِك}\\\sout{\RL{ذَانِلَك، تَانِلَك}}}
      }}
  \end{center}
\end{preview}
\end{document}