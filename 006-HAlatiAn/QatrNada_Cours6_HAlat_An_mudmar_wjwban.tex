% Generated by LaTreeX, <https:\\lautgesetz.com/latreex/>
% !TEX root = ../000_Complet/A4_006.tex
\documentclass{article}
\usepackage[a4paper,margin=1in,landscape]{geometry}
\usepackage[active,tightpage,xetex]{preview}
\usepackage[tuenc]{fontspec}
\usepackage{pstricks,pst-node,pst-tree}
\usepackage{varwidth,realscripts}
\usepackage{graphicx}
\usepackage{bidi}
\usepackage{titling}
\defaultfontfeatures{Mapping=tex-text}
\setmainfont{Amiri-Bold}[Script=Arabic]
\pagestyle{empty}
\psset{showbbox=false,
       treemode=D,
       linewidth=0.3pt,
       treesep=2em,
       levelsep=32ex,
       arrows=->,
       edge=\ncangles,
       angleA=-90,
       angleB=90,
       armA=3ex,
       armB=1ex}
\newcommand{\LFTw}[2]{%
\TR[name=#1,ref=t]{\psframebox[linestyle=none,framesep=3pt]{%
  \begin{varwidth}{6em}\center #2\end{varwidth}}}
}
\newcommand{\LFTwl}[2]{%
\TR[ref=#1]{\psframebox[linestyle=none,framesep=3pt]{%
  \begin{varwidth}{15em}\center #2\end{varwidth}}}
}
\newcommand{\LFTwsolid}[2]{%
\TR[ref=#1]{\psframebox[linestyle=solid,framesep=3pt]{%
  \begin{varwidth}{30ex}\center #2\end{varwidth}}}
}
\newcommand{\LFTr}[2]{%
\TR[ref=#1]{%
  \psframebox[linestyle=none,framesep=2pt]{#2}}
}

\newfontfamily{\exemplefont}{Amiri}[Scale=0.9,Script=Arabic]
\newfontfamily{\mainfont}{Amiri-Bold}[Script=Arabic]
\newcommand{\exemple}[1]{%
  \fontspec{Amiri}[Scale=0.9,Script=Arabic]
  \RL{#1}
}
\newcommand{\annot}[1]{%
  \fontspec{Amiri}[Scale=0.7,Script=Arabic]
  \RL{#1}
}
\newcommand{\emphaseVV}[1]{\char"200D\textcolor{black!40}{\char"200D #1\char"200D}{\char"200D}}
\newcommand{\emphaseVF}[1]{\char"200D\textcolor{black!40}{\char"200D #1}}
\newcommand{\emphaseFV}[1]{\textcolor{black!40}{#1\char"200D}{\char"200D}}
\newcommand{\emphaseFF}{\textcolor{black!40}}

\def\pstreehooki{\psset{thislevelsep=*0pt}}
\def\pstreehookii{\psset{thislevelsep=3ex}}
\def\pstreehookiii{\psset{thislevelsep=*0pt}}
\def\pstreehookv{\psset{thislevelsep=*0pt}}
\def\pstreehookvii{\psset{thislevelsep=*0pt}}
\def\pstreehookix{\psset{thislevelsep=*0pt}}
\def\pstreehookxi{\psset{thislevelsep=*0pt}}
\def\pstreehookxiii{\psset{thislevelsep=*0pt}}
\def\pstreehookxv{\psset{thislevelsep=*0pt}}

\pretitle{}\posttitle{}\title{}
\preauthor{}\postauthor{}
\author{Cours 6 - \RL{حالات أَن}}
\predate{}\postdate{}\date{}

\begin{document}
\begin{preview}
  \begin{center}
    
    \large %small,normalsize,large,Large,LARGE,huge,HUGE

    %pstree
    % \\ \begin{flushright}{\exemplefont  } \end{flushright} 
    \pstree{\LFTw{t}{\RL{مُضمَر وجوباً}}}{\pstree{\Tp[edge=none]}{%
        \pstree{\LFTw{t}{\RL{بَعدَ}}}{\pstree{\Tp[edge=none]}{%
            \LFTwl{t}{\RL{أَو بِمَعنَى إِلى/إِلاَّ \\[1\baselineskip]
                \mainfont{مَعنَى إِلى:}\exemplefont{لَأُطِيعَنَّ أَو يغْف\emphaseVF{رَ}لِي} \\
                \mainfont{مَعنَى إِلَّا:}\exemplefont{لَأَقْتُلَنَّ العَدُوَّ أَو يسلِ\emphaseVF{مَ}}
              }}
            \LFTwl{t}{\RL{وَاو المَعِيَّة \\[1\baselineskip] \exemplefont{اجتَهِدْ وَتنجَ\emphaseVF{حَ}}}}
            \LFTwl{t}{فَاء السَّبَبِيّة \\ \exemplefont{
                الطَّلَب\\ \begin{flushright}\RL{
                    ١.الأَمر: اِجتَهِدْ فَتَنْجَ\emphaseVF{حَ}\\
                    ٢.النَّهِي: لَا تَظْلِمْ فَتُظلَ\emphaseVF{مَ} \\
                    ٣.الاِستِفهَام: هَل تَزُورنِي فَأُكْرِ\emphaseFV{مَ}كَ \\
                    ٤.التَّمَنِّي: لَيْتَكَ مَوْجُودٌ فَأَزُو\emphaseFF{رَ}كَ \\
                    ٥.الرَّجَاء: لَعَلَّ الاِمتِحَانَ سهلٌ فَأَفُو\emphaseFF{زَ}\\
                    ٦.الدُعَاء: اللّهُمَّ اِهْدِنِي فَأَعمَ\emphaseVF{لَ} الخَير\\
                    ٧.العَرضُ (طَلَب بِرَفق): ألا تَزُورُنِي فَأُكْرِ\emphaseFV{مَ}كَ\\
                    ٨.التَّحضِيضُ: هَلاَّ تُْسلِمُ فَتَدْخُ\emphaseVF{لَ} الَجَنَّة\\
                  }\end{flushright}
              }}
            \LFTwl{t}{\RL{حَتَّى (حَرف جَر، غَايَة وَ/أَو تَعلِيل) \\[1\baselineskip]
                \mainfont{غَايَة:}\exemplefont{لا يُؤمِن أحدكم حتى يح\emphaseVF{بَّ} لِأخيه} \\
                \mainfont{تَعلِيل:}\exemplefont{أسلِمْ حتى تَدخ\emphaseVF{لَ} الجنّةَ}
              }}
            \LFTw{t}{\RL{لام الجُحُود \\[1\baselineskip]\exemplefont{ مَا كَانَ محمّد لِيَكذ\emphaseVF{بَ} \\ لَم يَكُنْ محمّد لِيَكذ\emphaseVF{بَ}}}}
          }}
      }}
  \end{center}
\end{preview}
\end{document}