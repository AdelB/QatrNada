% Generated by LaTreeX, <https:\\lautgesetz.com/latreex/>
% !TEX root = ../000_Complet/A4_006.tex
\documentclass{article}
\usepackage[a4paper,margin=1in,portrait]{geometry}
\usepackage[active,tightpage,xetex]{preview}
\usepackage[tuenc]{fontspec}
\usepackage{pstricks,pst-node,pst-tree}
\usepackage{varwidth,realscripts}
\usepackage{graphicx}
\usepackage{bidi}
\usepackage{titling}
\defaultfontfeatures{Mapping=tex-text}
\setmainfont{Amiri-Bold}[Script=Arabic]
\pagestyle{empty}
\psset{showbbox=false,
       treemode=D,
       linewidth=0.3pt,
       treesep=2em,
       levelsep=20ex,
       arrows=->,
       edge=\ncangles,
       angleA=-90,
       angleB=90,
       armA=3ex,
       armB=1ex}
\newcommand{\LFTw}[2]{%
\TR[name=#1,ref=t]{\psframebox[linestyle=none,framesep=3pt]{%
  \begin{varwidth}{15em}\center #2\end{varwidth}}}
}
\newcommand{\LFTwl}[2]{%
\TR[ref=#1]{\psframebox[linestyle=none,framesep=3pt]{%
  \begin{varwidth}{15em}\center #2\end{varwidth}}}
}
\newcommand{\LFTwsolid}[2]{%
\TR[ref=#1]{\psframebox[linestyle=solid,framesep=3pt]{%
  \begin{varwidth}{30ex}\center #2\end{varwidth}}}
}
\newcommand{\LFTr}[2]{%
\TR[ref=#1]{%
  \psframebox[linestyle=none,framesep=2pt]{#2}}
}

\newfontfamily{\exemplefont}{Amiri}[Scale=0.9,Script=Arabic]
\newcommand{\exemple}[1]{%
  \fontspec{Amiri}[Scale=0.9,Script=Arabic]
  \RL{#1}
}
\newcommand{\annot}[1]{%
  \fontspec{Amiri}[Scale=0.7,Script=Arabic]
  \RL{#1}
}
\newcommand{\emphaseVV}[1]{\char"200D\textcolor{black!40}{\char"200D #1\char"200D}{\char"200D}}
\newcommand{\emphaseVF}[1]{\char"200D\textcolor{black!40}{\char"200D #1}}
\newcommand{\emphaseFV}[1]{\textcolor{black!40}{#1\char"200D}{\char"200D}}
\newcommand{\emphaseFF}{\textcolor{black!40}}

\def\pstreehooki{\psset{thislevelsep=*0pt}}
\def\pstreehookiii{\psset{thislevelsep=*0pt}}
\def\pstreehookv{\psset{thislevelsep=*0pt}}
\def\pstreehookvii{\psset{thislevelsep=*0pt}}
\def\pstreehookix{\psset{thislevelsep=*0pt}}
\def\pstreehookxi{\psset{thislevelsep=*0pt}}
\def\pstreehookxiii{\psset{thislevelsep=*0pt}}
\def\pstreehookxv{\psset{thislevelsep=*0pt}}

\pretitle{}\posttitle{}\title{}
\preauthor{}\postauthor{}
\author{Cours 6 - \RL{حالات أَن}}
\predate{}\postdate{}\date{}

\begin{document}
\begin{preview}
  \begin{center}
    
    \normalsize %small,normalsize,large,Large,LARGE,huge,HUGE

    %pstree
    % \\ \begin{flushright}{\exemplefont  } \end{flushright} 

    \pstree{\LFTw{t}{\RL{مُضمَر جَوازاً}}}{\pstree{\Tp[edge=none]}{%
        \pstree{\LFTw{t}{\RL{بَعدَ لام الجَر}}}{\pstree{\Tp[edge=none]}{%
            \LFTw{t}{\RL{الزَّائِدة \\ (بِمَعنَى التَّوكِيد)   \\ \begin{flushright}{\exemplefont إِنَّمَا يُرِيدُ مُحَمَد لِ(أَن)يضرِ\emphaseFF{بَ} زَِيداً} \end{flushright}}}
            \LFTw{t}{\RL{الصَّيرُورَة \\[1\baselineskip] \begin{flushright}{\exemplefont حَفِظْتَ القُرآن لِ(أَن)تَكُو\emphaseFF{نَ} كَافِراً فِي آخِر حَيَاتِك} \end{flushright}}}
            \LFTw{t}{\RL{التَّعلِيل \\ \begin{flushright}{\exemplefont خَرَجتُ مِنَ الفَصلِ لِ(أَن)أَشرَ\emphaseFF{بَ} المَاءَ  \\ (لِأَنْ أَشرُ\emphaseFF{بَ} أَو لِشُربِ الماءَ) } \end{flushright}}}
          }}
        \LFTw{t}{\RL{ بَعدَ عاطِف مَسبُوق بِاسم خالص مِن مَعنَى الفِعل \smallskip (كَالمَصدَر) \\ (أَو الاِسم الجامِد المحض)\\ \begin{flushright}{\exemplefont قِراءَةُ القُرآنِ و(أَن)تَحفَ\emphaseVV{ظَ}هُ خَيرٌ لَكَ مِن أَنْ تَلعَ\emphaseVF{بَ} (مِنَ اللَّعِبِ)} \end{flushright}}}
      }}

    \pstree{\LFTw{t}{\RL{ظاهِر وجوباً}}}{\pstree{\Tp[edge=none]}{%
        \pstree{\LFTw{t}{\RL{بَِينَ لام الجَر و لا}}}{\pstree{\Tp[edge=none]}{%
            \LFTw{t}{\RL{لا الزَّائِدة \\ (بِمَعنَى التَّوكِيد) \\\begin{flushright}{\exemplefont  ذَهَبْتُ إِلى المَسجِدِ لِئلَّا أُصَلِ\emphaseVF{يَّ} } \end{flushright} }}
            \LFTw{t}{\RL{لا النَّافِية  \\ \begin{flushright}{\exemplefont ضَرَبْتُ مُحَمَّداً لِئَلَّا يَخرُ\emphaseFF{جَ} مِنَ البَيتِ } \end{flushright} }}
          }}
      }}
  \end{center}
\end{preview}
\end{document}