% Generated by LaTreeX, <https:\\lautgesetz.com/latreex/>
% !TEX root = ../000_Complet/A4_005.tex
\documentclass{article}
\usepackage[a4paper,margin=1in,landscape]{geometry}
\usepackage[active,tightpage,xetex]{preview}
\usepackage[tuenc]{fontspec}
\usepackage{pstricks,pst-node,pst-tree}
\usepackage{varwidth,realscripts}
\usepackage{graphicx}
\usepackage{bidi}
\usepackage{titling}
\defaultfontfeatures{Mapping=tex-text}
\setmainfont{Amiri-Bold}[Script=Arabic]
\pagestyle{empty}
\psset{showbbox=false,
       treemode=D,
       linewidth=0.3pt,
       treesep=2em,
       levelsep=22ex,
       arrows=->,
       edge=\ncangles,
       angleA=-90,
       angleB=90,
       armA=3ex,
       armB=3ex}
\newcommand{\LFTw}[2]{%
  \TR[ref=#1]{\psframebox[linestyle=none,framesep=3pt]{%
    \begin{varwidth}{9em}\center #2\end{varwidth}}}
}\newcommand{\LFTwl}[2]{%
\TR[ref=#1]{\psframebox[linestyle=none,framesep=3pt]{%
  \begin{varwidth}{15em}\center #2\end{varwidth}}}
}
\newcommand{\LFTwsolid}[2]{%
  \TR[ref=#1]{\psframebox[linestyle=solid,framesep=3pt]{%
    \begin{varwidth}{30ex}\center #2\end{varwidth}}}
}
\newcommand{\LFTr}[2]{%
  \TR[ref=#1]{%
    \psframebox[linestyle=none,framesep=2pt]{#2}}
}
\newfontfamily{\exemplefont}{Amiri}[Scale=0.9,Script=Arabic]
\newcommand{\exemple}[1]{%
  \fontspec{Amiri}[Scale=0.9,Script=Arabic]
  \RL{#1}
}
\newcommand{\annot}[1]{%
  \fontspec{Amiri}[Scale=0.7,Script=Arabic]
  \RL{#1}
}

\def\pstreehooki{\psset{thislevelsep=*0pt}}
\def\pstreehookiii{\psset{thislevelsep=*0pt}}
\def\pstreehookv{\psset{thislevelsep=*0pt}}
\def\pstreehookvii{\psset{thislevelsep=*0pt}}
\def\pstreehookix{\psset{thislevelsep=*0pt}}
\def\pstreehookxi{\psset{thislevelsep=*0pt}}
\def\pstreehookxiii{\psset{thislevelsep=*0pt}}
\def\pstreehookxv{\psset{thislevelsep=*0pt}}

\pretitle{}\posttitle{}\title{}
\preauthor{}\postauthor{}
\author{Cours 5 - \RL{إعراب الفِعل المُضارِع}}
\predate{}\postdate{}\date{}

\begin{document}
\begin{preview}
  \begin{center}
    
    \large %small,normalsize,large,Large,LARGE,huge,HUGE
    %pstree
    \pstree{\LFTw{t}{\RL{النَّواصِب}}}{\pstree{\Tp[edge=none]}{%
    \LFTw{t}{\RL{أَنْ\\{\exemplefont حَرف نَصبٍ وَ مَصدَرٍ}}}
    \pstree{\LFTw{t}{\RL{إِذَن ( إِذا ) \\{\exemplefont حَرف نَصبٍ وَ جَوَابٍ وَ جَزَاءٍ}}}}{\pstree{\Tp[edge=none]}{%
    \LFTw{t}{\RL{ الشُروط لِيَنصِبَ \\[1\baselineskip]- المُضَارِع فِي المُستقبَل\\- فِي صَدرِ الجُملَة\\- لا فاصل بِيَنها وَ الفِعل إلَّا بِالقسم}}
    \LFTw{t}{\RL{مثل\\سَأَزُورُكَ غداً\\{\exemplefont  إِذنْ أكرمَك \\إِنَّ إِذنْ أكرمُك \\ إِذنْ وَاللهِ أكرمَك}\\[1\baselineskip] أَقُولُ لَكَ الحَق \\{\exemplefont إِذنْ أَصدقُك} }}
    }}
    \pstree{\LFTwl{t}{\RL{كَي \\{\exemplefont حَرف نَصبٍ وَ مَصدرٍ } \\ كَي + فِعل مُضارِع = مَصدَر \\{\exemplefont خَرَجتُ مِنَ الفَصلِ لِكَِي أَشرُبَ المَاَءَ }\\ المَصدَر المُؤَول (كَي أَشرُبَ) فِي مَحَل الجَر \\[1\baselineskip]{\exemplefont خَرَجتُ مِنَ الفَصلِ كَي أَشرُبَ}}}}{\pstree{\Tp[edge=none]}{%
    \LFTw{t}{\RL{كَي (أَن) \\{\exemplefont كَي = حَرف جَر \\  أَن = حَرف نَصب}}}
    \LFTw{t}{\RL{(لِ) كَي  \\{\exemplefont لِ = حَرف جَر \\ كَي = حَرف نَصب}}}
    }}
    \LFTw{t}{\RL{لَنْ \\{\exemplefont حَرف نَصبٍ و نَفِي و استِقبَالٍ}}}
    }}
  \end{center}
\end{preview}
\end{document}