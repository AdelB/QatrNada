% Generated by LaTreeX, <https:\\ \exemplefontlautgesetz.com/latreex/>
% TEX root = ../000_Complet/A4_008_6.tex
\documentclass{article}
\usepackage[a4paper,margin=1in,landscape]{geometry}
\usepackage[active,tightpage,xetex]{preview}
\usepackage[tuenc]{fontspec}
\usepackage{pstricks,pst-node,pst-tree}
\usepackage{varwidth,realscripts}
\usepackage{graphicx}
\usepackage{bidi}
\usepackage{titling}
\defaultfontfeatures{Mapping=tex-text}
\setmainfont{Amiri-Bold}[Script=Arabic,Scale=1.2]
\pagestyle{empty}
\psset{showbbox=false,
       treemode=D,
       linewidth=0.3pt,
       treesep=1em,
       levelsep=5em,
       arrows=->,
       edge=\ncangles,
       angleA=-90,
       angleB=90,
       armA=3em,
       armB=2em}
\newcommand{\LFTw}[2]{%
\Tr[name=#1,ref=t]{\psframebox[linestyle=none,framesep=3pt]{%
  \begin{varwidth}{10em}\center #2\end{varwidth}}}
}
\newcommand{\LFTwl}[2]{%
\Tr[ref=#1]{\psframebox[linestyle=none,framesep=3pt]{%
  \begin{varwidth}{18em}\center #2\end{varwidth}}}
}
\newcommand{\LFTwsolid}[2]{%
\Tr[ref=#1]{\psframebox[linestyle=solid,framesep=3pt]{%
  \begin{varwidth}{30ex}\center #2\end{varwidth}}}
}
\newcommand{\LFTr}[2]{%
\Tr[ref=#1]{%
  \psframebox[linestyle=none,framesep=2pt]{#2}}
}

\newfontfamily{\exemplefont}{Amiri}[Scale=1,Script=Arabic]
\newfontfamily{\moyennefont}{Amiri-Bold}[Scale=1,Script=Arabic]
\newcommand{\exemple}[1]{%
  \fontspec{Amiri}[Scale=1,Script=Arabic]
  \RL{#1}
}
\newcommand{\annot}[1]{%
  \fontspec{Amiri}[Scale=1,Script=Arabic]
  \RL{#1}
}

\newcommand{\emphaseVV}[1]{\char"200D\textcolor{black!40}{\char"200D #1\char"200D}{\char"200D}}
\newcommand{\emphaseVF}[1]{\char"200D\textcolor{black!40}{\char"200D #1}}
\newcommand{\emphaseFV}[1]{\textcolor{black!40}{#1\char"200D}{\char"200D}}
\newcommand{\emphaseFF}{\textcolor{black!40}}

\def\pstreehooki{\psset{thislevelsep=*0pt}}
\def\pstreehookiii{\psset{thislevelsep=*0pt}}
\def\pstreehookv{\psset{thislevelsep=*0pt}}
\def\pstreehookvii{\psset{thislevelsep=*0pt}}
\def\pstreehookix{\psset{thislevelsep=*0pt}}
\def\pstreehookxi{\psset{thislevelsep=*0pt}}
\def\pstreehookxiii{\psset{thislevelsep=*0pt}}
\def\pstreehookxv{\psset{thislevelsep=*0pt}}

\pretitle{}\posttitle{}\title{\RL{اِسم مَعرِفَة}}
\preauthor{}\postauthor{}
\author{}
\predate{}\postdate{}\date{}

\begin{document}
\begin{preview}
  \begin{center}

    \small %small,normalsize,large,Large,LARGE,huge,HUGE
    %pstree
    % \\ \exemplefont \begin{flushright}{\exemplefont  } \end{flushright}  \emphaseVV{}
    \LFTwl{t}{\RL{المفعُول لَهُ /المَفعُول لِأَجلِهِ\\ كُلُّ مَصدَرٍ مُعَلِّل لِحَدَثِ مُشَارِكٍ لَهُ فِي الزَمَان وَ الفَاعِل \\ \exemplefont قُمتُ \emphaseFF{إِجلاَلاً} لَكَ \\ \moyennefont زَمَنُ الإِجلاَل وَ القِيَام وَاحِد، و فَاعِل القِيَام وَ فَاعِل الإِجلاَل وَاحِد \\ \exemplefont يَجْعَلُونَ أَصَابِعَهُمْ فِي آذَانِهِم مِّنَ الصَّوَاعِقِ \emphaseFF{حَذَرَ الْمَوْتِ}}}

    \begin{center}\line(1,0){450}\end{center}

    \pstree{\LFTw{t}{\RL{المَفعُول فِيهِ / الظَرف \\ \moyennefont كُلُّ اسمِ زَمَان أَو مَكَان سُلِّطَ عَلَِيهِ عَامِل عَلى مَعنَى (فِي) بِاطِّرَاد}}}{\pstree{\Tp[edge=none]}{%
        \pstree{\LFTw{t}{\RL{اسماء المَكَان\\ \moyennefont مُبهَم}}}{\pstree{\Tp[edge=none]}{%
            \LFTw{t}{\RL{مَا كَانَ مَصُوغاً مِن مَصدَرٍ عَامِلِهِ\\ \exemplefont  جَلَستُ \emphaseFF{مَجلِسَ} زَيد}}
            \LFTw{t}{\RL{اسمَاء مَقَادِير المَسَاحَات\\ \exemplefont  سِرتُ \emphaseFF{فَرسَخاً} \\ المَيل \\ البَرِيد}}
            \LFTw{t}{\RL{اسمَاء الجِهَات السِتة\\ \exemplefont  فُوق/تَحت (أَسفَلَ)/أَمَام/خَلف(وَرَاء)/يَمِين/يَسَار\\وَالرَّكْبُ \emphaseFF{أَسْفَلَ} مِنكُمْ}}
          }}
        %\LFTw{t}{\RL{اسماء الزَمَان\\ \moyennefont تَقبَل النَصب عَلى الظَرفِيَّة\\ \exemplefont  صُمتُ \emphaseFF{يَومَ} الخَمِيس}}
        \pstree{\LFTw{t}{\RL{اسماء الزَمَان\\ \moyennefont تَقبَل النَصب عَلى الظَرفِيَّة}}}{\pstree{\Tp[edge=none]}{%
            \LFTw{t}{\RL{اسم زَمَان مُبهَم \\ \moyennefont لاَ يَقَع جَوَاباً لِشَيءٍ\\حِينَ/وَقتَ/مُدَّةَ}}
            \LFTw{t}{\RL{اسمُ زَمَان مَعدُود \\ \moyennefont جَوَاباً لِ(كَم) \\ \exemplefont  كَم جَلَستَ فِي مَكَّةَ ؟ جَلَستُ \emphaseFF{أُسبُعاً}}}
            \LFTw{t}{\RL{اسم زَمَان مُختَص \\ \moyennefont جَوَاباً لِ(مَتَى) \\ \exemplefont  مَتَى صُمتَ ؟ \emphaseFF{يَومَ} الخَمِيس}}
          }}
      }}

  \end{center}
\end{preview}
\end{document}