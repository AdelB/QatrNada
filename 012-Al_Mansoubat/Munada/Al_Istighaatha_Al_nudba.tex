% Generated by LaTreeX, $\leftarrow$https:\\ \exemplefontlautgesetz.com/latreex/>
% TEX root = ../000_Complet/A4_008_6.tex
\documentclass{article}
\usepackage[a4paper,margin=1in,landscape]{geometry}
\usepackage[active,tightpage,xetex]{preview}
\usepackage[tuenc]{fontspec}
\usepackage{pstricks,pst-node,pst-tree}
\usepackage{varwidth,realscripts}
\usepackage{graphicx}
\usepackage{bidi}
\usepackage{titling}
\defaultfontfeatures{Mapping=tex-text}
\setmainfont{Amiri-Bold}[Script=Arabic,Scale=1.2]
\pagestyle{empty}
\psset{showbbox=false,
       treemode=D,
       linewidth=0.3pt,
       treesep=1em,
       levelsep=2em,
       arrows=->,
       edge=\ncangles,
       angleA=-90,
       angleB=90,
       armA=1em,
       armB=1em}
\newcommand{\LFTw}[2]{%
\Tr[name=#1,ref=t]{\psframebox[linestyle=none,framesep=3pt]{%
  \begin{varwidth}{10em}\center #2\end{varwidth}}}
}
\newcommand{\LFTwl}[2]{%
\Tr[ref=#1]{\psframebox[linestyle=none,framesep=3pt]{%
  \begin{varwidth}{15em}\center #2\end{varwidth}}}
}
\newcommand{\LFTwsolid}[2]{%
\Tr[ref=#1]{\psframebox[linestyle=solid,framesep=3pt]{%
  \begin{varwidth}{30ex}\center #2\end{varwidth}}}
}
\newcommand{\LFTr}[2]{%
\Tr[ref=#1]{%
  \psframebox[linestyle=none,framesep=2pt]{#2}}
}

\newfontfamily{\exemplefont}{Amiri}[Scale=1,Script=Arabic]
\newfontfamily{\moyennefont}{Amiri-Bold}[Scale=1,Script=Arabic]
\newcommand{\exemple}[1]{%
  \fontspec{Amiri}[Scale=1,Script=Arabic]
  \RL{#1}
}
\newcommand{\annot}[1]{%
  \fontspec{Amiri}[Scale=1,Script=Arabic]
  \RL{#1}
}

\newcommand{\emphaseVV}[1]{\char"200D\textcolor{black!40}{\char"200D #1\char"200D}{\char"200D}}
\newcommand{\emphaseVF}[1]{\char"200D\textcolor{black!40}{\char"200D #1}}
\newcommand{\emphaseFV}[1]{\textcolor{black!40}{#1\char"200D}{\char"200D}}
\newcommand{\emphaseFF}{\textcolor{black!40}}

\def\pstreehooki{\psset{thislevelsep=*0pt}}
\def\pstreehookiii{\psset{thislevelsep=*0pt}}
\def\pstreehookv{\psset{thislevelsep=*0pt}}
\def\pstreehookvii{\psset{thislevelsep=*0pt}}
\def\pstreehookix{\psset{thislevelsep=*0pt}}
\def\pstreehookxi{\psset{thislevelsep=*0pt}}
\def\pstreehookxiii{\psset{thislevelsep=*0pt}}
\def\pstreehookxv{\psset{thislevelsep=*0pt}}

\pretitle{}\posttitle{}\title{\RL{اِسم مَعرِفَة}}
\preauthor{}\postauthor{}
\author{}
\predate{}\postdate{}\date{}

\begin{document}
\begin{preview}
  \begin{center}

    \normalsize %small,normalsize,large,Large,LARGE,huge,HUGE
    %pstree
    % \\ \exemplefont \begin{flushright}{\exemplefont  } \end{flushright}  \emphaseVV{}
    \pstree{\LFTw{t}{\RL{الاِستِغَاثَة \\ \moyennefont هيَ نِدَاء مَن يَخلُصُ مِن شِدَّةٍ وَاقِعة أَو يُعَيِّنُ عَلَى دَفعِهَا قَبلَ وُقُوعِهَا }}}{\pstree{\Tp[edge=none]}{%
        \pstree{\LFTw{t}{\RL{استِعمَالاَتُهَا}}}{\pstree{\Tp[edge=none]}{%
            \pstree{\LFTw{t}{\RL{حَذف اللاَّم}}}{\pstree{\Tp[edge=none]}{%
                \LFTwl{t}{\RL{لاَ يُزَاد في آخِر المُستَغَاث بِهِ \emphaseFF{أَلِف}\\ \moyennefont وَ هَذَا الأَقَلُّ استِعمَالاً\\ \exemplefont  يَا خَالِدُ لِعَمرو}}
                \LFTwl{t}{\RL{يُزَاد في آخِر المُستَغَاث بِهِ أَلِف\\ \exemplefont يَا عَالِم\emphaseVF{ا} لِلجاهِل}}
              }}
            \LFTw{t}{\RL{الأَصل}}
          }}
        \pstree{\LFTw{t}{\RL{أَركَانُهَا}}}{\pstree{\Tp[edge=none]}{%
            \pstree{\LFTw{t}{\RL{المُستَغَاث لَهُ}}}{\pstree{\Tp[edge=none]}{%
                \pstree{\LFTw{t}{\RL{مَجرُوُرٌ بِ}}}{\pstree{\Tp[edge=none]}{%
                    \LFTw{t}{\RL{مِن\\ \exemplefont يَا لَلقاضِي \emphaseFF{مِن} شَاهِد الزُور}}
                    \LFTw{t}{\RL{لَام مَكسُورَة\\ \exemplefont يَا لَلنَاس \emphaseFV{لِ}لغَرِيق}}
                  }}
              }}
            \pstree{\LFTw{t}{\RL{المُستَغَاث بِهِ}}}{\pstree{\Tp[edge=none]}{%
                \LFTwl{t}{\RL{إِلاَّ إِذَا عُطِفَ عَلَيهِ مُستَغَاث آخَر\\ \exemplefont يَا لَلعُلَمَاءِ وَ \emphaseFV{لِ}لمُصلِحين لِلشَبَاب}}
                \LFTw{t}{\RL{مَجرُورٌ بِلَامٍ مَفتُوحَة\\ \moyennefont غَالِباً \\\exemplefont يَا \emphaseFV{لَ}لنَاس لِلغَرِيق}}
              }}
            \LFTw{t}{\RL{حَرف النِّدَاء\\ \exemplefont  يَا}}
          }}
      }}
    \begin{center}\line(1,0){450}\end{center}
    \pstree{\LFTwl{t}{\RL{النُّدبَة \\ نِدَاء المُتَفَجَّع عَليهِ لِفَقدِهِ أَو المُتَوَجَّع مِنهُ لِكَونِهِ مَحَلَّ أَلَم }}}{\pstree{\Tp[edge=none]}{%
        \pstree{\LFTw{t}{\RL{أَحكَام المَندُوب}}}{\pstree{\Tp[edge=none]}{%
            \pstree{\LFTw{t}{\RL{إِلحَاقُهُ بِ(هَاء) بَعدَ الأَلِف\\ \exemplefont وَا زَيدَاه}}}{\pstree{\Tp[edge=none]}{%
                \LFTw{t}{\RL{تُكسَرُ فِي التِقَاء السَاكِنَينِ\\ \exemplefont وَا زَيدَاهِ}}
                \LFTw{t}{\RL{تُضَمُّ كَالضَّمِير\\ \exemplefont وَا زَيدَاهُ.}}
                \LFTw{t}{\RL{تُحذَف فِي الوَصل\\ \exemplefont وَا زَيدَا ...}}
              }}
            \LFTw{t}{\RL{فِي آخِرِهِ أَلِف\\ \exemplefont وَا زَيدَا}}
            \LFTw{t}{\RL{مَنصُوب\\ \exemplefont  وَا عَبدَ اللَّهِ}}
            \LFTw{t}{\RL{مَبني عَلى الضَّم\\ \exemplefont  وَا زَيدُ}}
          }}
        \pstree{\LFTw{t}{\RL{أَدَوَات النُّدبَة}}}{\pstree{\Tp[edge=none]}{%
            \LFTw{t}{\RL{يَا}}
            \LFTw{t}{\RL{وَا\\ \moyennefont غَالِباً}}
          }}
      }}
    \SpecialCoor
    \psline[linewidth=0.5pt]{->}(!\psGetNodeCenter{I1} I1.x 1 sub I1.y )(!\psGetNodeCenter{I2} I2.x 1.2 add I2.y 0.5 sub)
  \end{center}
\end{preview}
\end{document}