% Generated by LaTreeX, <https:\\ \exemplefontlautgesetz.com/latreex/>
% TEX root = ../000_Complet/A4_008_6.tex
\documentclass{article}
\usepackage[a3paper,margin=1in,landscape]{geometry}
\usepackage[active,tightpage,xetex]{preview}
\usepackage[tuenc]{fontspec}
\usepackage{pstricks,pst-node,pst-tree}
\usepackage{varwidth,realscripts}
\usepackage{graphicx}
\usepackage{bidi}
\usepackage{titling}
\defaultfontfeatures{Mapping=tex-text}
\setmainfont{Amiri-Bold}[Script=Arabic,Scale=1.2]
\pagestyle{empty}
\psset{showbbox=false,
       treemode=D,
       linewidth=0.3pt,
       treesep=1em,
       levelsep=2em,
       arrows=->,
       edge=\ncangles,
       angleA=-90,
       angleB=90,
       armA=1em,
       armB=1em}
\newcommand{\LFTw}[2]{%
\Tr[name=#1,ref=t]{\psframebox[linestyle=none,framesep=3pt]{%
  \begin{varwidth}{10em}\center #2\end{varwidth}}}
}
\newcommand{\LFTwl}[2]{%
\Tr[ref=#1]{\psframebox[linestyle=none,framesep=3pt]{%
  \begin{varwidth}{15em}\center #2\end{varwidth}}}
}
\newcommand{\LFTwsolid}[2]{%
\Tr[ref=#1]{\psframebox[linestyle=solid,framesep=3pt]{%
  \begin{varwidth}{30ex}\center #2\end{varwidth}}}
}
\newcommand{\LFTr}[2]{%
\Tr[ref=#1]{%
  \psframebox[linestyle=none,framesep=2pt]{#2}}
}

\newfontfamily{\exemplefont}{Amiri}[Scale=1,Script=Arabic]
\newfontfamily{\moyennefont}{Amiri-Bold}[Scale=1,Script=Arabic]
\newcommand{\exemple}[1]{%
  \fontspec{Amiri}[Scale=1,Script=Arabic]
  \RL{#1}
}
\newcommand{\annot}[1]{%
  \fontspec{Amiri}[Scale=1,Script=Arabic]
  \RL{#1}
}

\newcommand{\emphaseVV}[1]{\char"200D\textcolor{black!40}{\char"200D #1\char"200D}{\char"200D}}
\newcommand{\emphaseVF}[1]{\char"200D\textcolor{black!40}{\char"200D #1}}
\newcommand{\emphaseFV}[1]{\textcolor{black!40}{#1\char"200D}{\char"200D}}
\newcommand{\emphaseFF}{\textcolor{black!40}}

\def\pstreehooki{\psset{thislevelsep=*0pt}}
\def\pstreehookiii{\psset{thislevelsep=*0pt}}
\def\pstreehookv{\psset{thislevelsep=*0pt}}
\def\pstreehookvii{\psset{thislevelsep=*0pt}}
\def\pstreehookix{\psset{thislevelsep=*0pt}}
\def\pstreehookxi{\psset{thislevelsep=*0pt}}
\def\pstreehookxiii{\psset{thislevelsep=*0pt}}
\def\pstreehookxv{\psset{thislevelsep=*0pt}}

\pretitle{}\posttitle{}\title{\RL{اِسم مَعرِفَة}}
\preauthor{}\postauthor{}
\author{}
\predate{}\postdate{}\date{}

\begin{document}
\begin{preview}
  \begin{center}

    \normalsize %small,normalsize,large,Large,LARGE,huge,HUGE
    %pstree
    % \\ \exemplefont \begin{flushright}{\exemplefont  } \end{flushright}  \emphaseVV{}
    \pstree{\LFTw{t}{\RL{أَحكَام تَابِع المُنَادَى}}}{\pstree{\Tp[edge=none]}{%
        \pstree{\LFTw{t}{\RL{إِذَا تَكَرَّرَ المُنَادَى المُفرَد مُضَافاً}}}{\pstree{\Tp[edge=none]}{%
            \LFTw{t}{\RL{فَتحُ الأَوَال وَ الثَانِي\\ {\exemplefont يَا زَيدَ زَيدَ اليعمُلاَت}\\الأَصل  \exemplefont يَا زَيدُ اليعمُلاَت زَيدَ اليعمُلاَت}}
            \pstree{\LFTw{t}{\RL{ضَمُّ الأَوَال\\فَالثَانِي\\ \exemplefont يَا زَيدُ زَيدَ اليعمُلاَت}}}{\pstree{\Tp[edge=none]}{%
                \LFTw{t}{\RL{مَفعُول بِهِ بِتَقدِير \emphaseFF{أَعنِي}\\ \exemplefont يَا زَيدُ \emphaseFF{أَعنِي} زَيدَ اليعمُلاَت}}
                \LFTw{t}{\RL{عَطف بَيَان}}
                \LFTw{t}{\RL{مُنَادَى حُذِفَ حَذف المُنَادَى \\ \exemplefont يَا زَيدُ \emphaseFF{يَا} زَيدَ اليعمُلاَت}}
              }}
          }}
        \pstree{\LFTw{t}{\RL{إِن كَانَ التَابِع \\\emphaseFF{أُعطِيَ مَا يَستَحِقُّ لَو كَانَ مُنَادَى}}}}{\pstree{\Tp[edge=none]}{%
            \LFTw{t}{\RL{نَسَقاً بِغَيرِ الأَلِف وَ اللَّام\\ \exemplefont  يَا زَيدُ وَ عَمرو\\ \exemplefont  يَا زَيدُ وَ أَبَا عَبدِ اللَّه}}
            \LFTw{t}{\RL{بَدَلاً بِغَيرِ الأَلِف وَ اللَّام \\  \exemplefont  يَا سَعِيدُ كُرزُ\\ \exemplefont  يَا سَعِيدُ أبَا عَبدِ اللَّه}}
          }}
        \LFTw{t}{\RL{النَعت التَابِع لِأَيِّ\\\emphaseFF{تَعَيَّن رَفعُهُ عَلى اللَّفظ}\\ \exemplefont  يَا أَيُّهَا}}
        \pstree{\LFTw{t}{\RL{إِذَا كَانَ المُنَادَى مَبنِياً\\وَ كَانَ تَابِعُهُ مُضَافاً وَ لَيسَ فِيهِ الَأَلِف وَ اللاَم\\\emphaseFF{تَعَيَّن النَصب عَلى المَحَل}}}}{\pstree{\Tp[edge=none]}{%
            \LFTw{t}{\RL{ \exemplefont  يَا زَيدُ صَاحِبَ عَمرو\\ \exemplefont  يَا زَيدُ أبَا عَبدِ اللَّه\\ \exemplefont يَا تَمِيمُ كُلَّهُم  \\ \exemplefont  يَا زَيدُ وَ أَبَا عَبدِ اللَّه}}
          }}
        \pstree{\LFTwl{t}{\RL{إِذَا كَانَ المُنَادَى مَبنِياً \\ وَ كَانَ تَابِعُهُ / وَ كَانَ مَعَ ذَلِكَ مُفرَداً أَو مُضَافاً وَ فيهِ الأَلِف وَ اللَّام}}}{\pstree{\Tp[edge=none]}{%
            \LFTw{t}{\RL{عَطفَ نُسُقاً بِاأَلِف وَ اللَّام\\ \exemplefont يَا زَيدُ وَ الضَّحَّكُ\\ \exemplefont يَا زَيدُ وَ الضَّحَّكَ}}
            \LFTw{t}{\RL{عَطفَ بَيَان\\ \exemplefont يَا سَعيدُ كُرزُ\\ \exemplefont يَا سَعيدُ كُرزاً}}
            \LFTw{t}{\RL{تَوكِداً\\ \exemplefont يَا تَمِيمُ أَجمَعُون\\ \exemplefont يَا تَمِيمُ أَجمَعِين}}
            \LFTw{t}{\RL{نَعتاً\\ \exemplefont يَا زَيدُ الظَّرِيفُ $\leftarrow$ \emphaseFF{تَبِعَهُ فِي اللَفظ}\\ \exemplefont يَا زَيدُ الظَّرِيفَ $\leftarrow$ \emphaseFF{تَبِعَهُ فِي المَحَل}}}
          }}
      }}
  \end{center}
\end{preview}
\end{document}