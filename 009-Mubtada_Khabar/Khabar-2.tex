% Generated by LaTreeX, <https:\\ \exemplefontlautgesetz.com/latreex/>
% !TEX root = ../000_Complet/A4_009.tex
\documentclass{article}
\usepackage[a3paper,margin=1in,landscape]{geometry}
\usepackage[active,tightpage,xetex]{preview}
\usepackage[tuenc]{fontspec}
\usepackage{pstricks,pst-node,pst-tree}
\usepackage{varwidth,realscripts}
\usepackage{graphicx}
\usepackage{ulem}
\usepackage{bidi}
\usepackage{titling}
\defaultfontfeatures{Mapping=tex-text}
\setmainfont{Amiri-Bold}[Script=Arabic,Scale=1.2]
\pagestyle{empty}
\psset{showbbox=false,
       treemode=D,
       linewidth=0.3pt,
       treesep=1em,
       levelsep=4em,
       arrows=->,
       edge=\ncangles,
       angleA=-90,
       angleB=90,
       armA=2em,
       armB=2em}
\newcommand{\LFTw}[2]{%
\Tr[name=#1,ref=t]{\psframebox[linestyle=none,framesep=3pt]{%
  \begin{varwidth}{10em}\center #2\end{varwidth}}}
}
\newcommand{\LFTwl}[2]{%
\Tr[ref=#1]{\psframebox[linestyle=none,framesep=3pt]{%
  \begin{varwidth}{15em}\center #2\end{varwidth}}}
}
\newcommand{\LFTwsolid}[2]{%
\Tr[ref=#1]{\psframebox[linestyle=solid,framesep=3pt]{%
  \begin{varwidth}{30ex}\center #2\end{varwidth}}}
}
\newcommand{\LFTr}[2]{%
\Tr[ref=#1]{%
  \psframebox[linestyle=none,framesep=2pt]{#2}}
}

\newfontfamily{\exemplefont}{Amiri}[Scale=1,Script=Arabic]
\newfontfamily{\moyennefont}{Amiri-Bold}[Scale=1,Script=Arabic]
\newcommand{\exemple}[1]{%
  \fontspec{Amiri}[Scale=1,Script=Arabic]
  \RL{#1}
}
\newcommand{\annot}[1]{%
  \fontspec{Amiri}[Scale=1,Script=Arabic]
  \RL{#1}
}

\newcommand{\emphaseVV}[1]{\char"200D\textcolor{black!40}{\char"200D #1\char"200D}{\char"200D}}
\newcommand{\emphaseVF}[1]{\char"200D\textcolor{black!40}{\char"200D #1}}
\newcommand{\emphaseFV}[1]{\textcolor{black!40}{#1\char"200D}{\char"200D}}
\newcommand{\emphaseFF}{\textcolor{black!40}}



\pretitle{}\posttitle{}\title{\RL{المُبتَدَأ و الخَبَر}}
\preauthor{}\postauthor{}
\author{}
\predate{}\postdate{}\date{}

\begin{document}
\begin{preview}
    \begin{center}
        
        \normalsize %small,normalsize,large,Large,LARGE,huge,HUGE
        %pstree
        % \\ \exemplefont \begin{flushright}{\exemplefont } \end{flushright}  \emphaseVV{}
        \pstree{\LFTw{t}{\RL{الخَبَر}}}{\pstree{\Tp[edge=none]}{%
                \Tn
                \pstree{\LFTw{t}{\RL{قَد يَتَقَدَّم \underline{الخَبَر}}}}{\pstree{\Tp[edge=none]}{%
                        \pstree{\LFTw{t}{\RL{وجُوباً}}}{\pstree{\Tp[edge=none]}{%
                                \LFTw{t}{\RL{الخَبَر مَحصور فِي المُبتَدَأ بِ \\ إِنَّمَا / إِلَّا \\ \exemplefont إِنَّمَا \underline{القَائِدُ} خَالِدٌ \exemplefont \\ \exemplefont مَا \underline{الهَادِي} إِلَّا الّلَه  }}
                                \LFTw{t}{\RL{الخَبَر لَهُ صَدرُ الجُملَة\\ \exemplefont \underline{\emphaseFF{أَينَ}} زَيدٌ}}
                                \LFTw{t}{\RL{فِي المُبتَدَأ ضَمير يعود\\  على بَعض الخَبَر \\ \exemplefont \underline{لِمجَاَلِسِ} العلم رُوَّادُ\emphaseFF{هَا}  }}
                                \LFTw{t}{\RL{المُبتَدَأ نَكِرَة \\ \exemplefont \underline{فِي الدَّارِ} \emphaseFF{رَجُلٌ}  }}
                            }}
                        \pstree{\LFTw{t}{\RL{جَوَازاً}}}{\pstree{\Tp[edge=none]}{%
                                \LFTw{t}{\RL{المُبتَدَأ مَعرِفَة \\ \exemplefont \underline{فِي الدَّارِ} \emphaseFF{زَيدٌ}  \\ \exemplefont ﴿\underline{سَلَامٌ} \emphaseFF{هِيَ}﴾  \\ ﴿وَ\underline{آيَةٌ} لَّهُمُ \emphaseFF{اللَّيْلُ}﴾}}
                            }}
                    }}
                \pstree{\LFTw{t}{\RL{قَد يَتَعَدَّد \underline{الخَبَر}}}}{\pstree{\Tp[edge=none]}{%
                        \LFTw{t}{\RL{لَفظاً لَا مَعنَى \\ \exemplefont هَذا البُرتُقَال \underline{حُلْوٌ حَامِض}\\ \exemplefont مُتوسط بَينَ الحَلاَوة وَ الحموضة  }}
                        \LFTw{t}{\RL{لَفظاً وَ مَعنَى \\ \exemplefont ﴿وَهُوَ \underline{الْغَفُورُ الْوَدُودُ}﴾  }}
                    }}
                \pstree{\LFTw{t}{\RL{وُجُوب حَذف \underline{الخَبَر}}}}{\pstree{\Tp[edge=none]}{%
                        \pstree{\LFTw{t}{\RL{وجُوباً}}}{\pstree{\Tp[edge=none]}{%
                                \LFTw{t}{\RL{بَعدَ الواو المُصاحِبَة الصَّريحَة (مَعَ) \\ \exemplefont كُلُّ رَجُلٍ \emphaseFF{وَ ضَيعَتُهُ}  \\ \exemplefont كُلُّ رَجُلٍ مَعَ ضَيعَتِهِ \exemplefont \\ \exemplefont الخَبَر مَحذُف : \underline{مَقترنَان}  }}
                                \LFTw{t}{\RL{قَبلَ الحَال المِمتَنِع كَونُهَا عَنِ المُبتَدَأ \\ \exemplefont ضَرْبِي زَيداً \emphaseFF{قَائِماً}  \\ \exemplefont الخَبَر مَحذُف : \\ \underline{إِذَا كَانَ} فِي المُستقبَل \\ \underline{إِذَ كَانَ} فِي المَاضي}}
                                \LFTw{t}{\RL{قَبلَ جَوَاب القَسَم الصَّريح \\ \exemplefont ﴿لَعَمْرُكَ \emphaseFF{إِنَّهُمْ لَفِي سَكْرَتِهِمْ يَعْمَهُونَ}﴾   \\ \exemplefont الخَبَر مَحذُف : \underline{قَسَمِي}  }}
                                \LFTw{t}{\RL{قَبلَ جَوَاب لَولاَ \\ \exemplefont ﴿لَوْلَا أَنتُمْ \emphaseFF{لَكُنَّا مُؤْمِنِينَ}﴾ \exemplefont \\ \exemplefont الخَبَر مَحذُف : \underline{صَدَدتُمُونَا}  }}
                            }}
                        \LFTw{t}{\RL{جَوَازاً}}
                    }}
                \pstree{\LFTwl{t}{\RL{إِذَا كَانَ المُبتَدَأ وَصفاً وَ يُسبَقُ \\ بِنَفِي أَو استِفهَام  فَلَيسَ لَهُ خَبَر لَكِن}}}{\pstree{\Tp[edge=none]}{%
                        \LFTw{t}{\RL{\underline{نَائِبُ فَاعِل سَدَّ مَسَدَّ الخَبَر}\\(\emphaseFF{مُبتَدَأ اسم مَفعُول})\\ \exemplefont مَا \emphaseFF{مَحبُوبٌ} \underline{المغتَابُون}   \\ \exemplefont أَ \emphaseFF{مُكرَمٌ} \underline{المَجدُون}   }}
                        \LFTw{t}{\RL{\underline{فَاعِل سَدَّ مَسَدَّ الخَبَر} \\(\emphaseFF{مُبتَدَأ اسم فَاعِل}) \\ \exemplefont مَا \emphaseFF{نَاجِحٌ} \underline{المُهمِلُون}  \\  \exemplefont أَ \emphaseFF{حَاضِرٌ} \underline{الطُّلاَب}   }}
                    }}
            }}
    \end{center}
\end{preview}
\end{document}