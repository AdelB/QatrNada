% Generated by LaTreeX, <https:\\lautgesetz.com/latreex/>
% !TEX root = ../000_Complet/A4_007.tex
\documentclass{article}
\usepackage[a4paper,margin=1in,landscape]{geometry}
\usepackage[active,tightpage,xetex]{preview}
\usepackage[tuenc]{fontspec}
\usepackage{pstricks,pst-node,pst-tree}
\usepackage{varwidth,realscripts}
\usepackage{graphicx}
\usepackage{bidi}
\usepackage{titling}
\defaultfontfeatures{Mapping=tex-text}
\setmainfont{Amiri-Bold}[Script=Arabic]
\pagestyle{empty}
\psset{showbbox=false,
       treemode=D,
       linewidth=0.3pt,
       treesep=2em,
       levelsep=15ex,
       arrows=->,
       edge=\ncangles,
       angleA=-90,
       angleB=90,
       armA=3ex,
       armB=1ex}
\newcommand{\LFTw}[2]{%
\TR[name=#1,ref=t]{\psframebox[linestyle=none,framesep=3pt]{%
  \begin{varwidth}{6em}\center #2\end{varwidth}}}
}
\newcommand{\LFTwl}[2]{%
\TR[ref=#1]{\psframebox[linestyle=none,framesep=3pt]{%
  \begin{varwidth}{15em}\center #2\end{varwidth}}}
}
\newcommand{\LFTwsolid}[2]{%
\TR[ref=#1]{\psframebox[linestyle=solid,framesep=3pt]{%
  \begin{varwidth}{30ex}\center #2\end{varwidth}}}
}
\newcommand{\LFTr}[2]{%
\TR[ref=#1]{%
  \psframebox[linestyle=none,framesep=2pt]{#2}}
}

\newfontfamily{\exemplefont}{Amiri}[Scale=0.9,Script=Arabic]
\newcommand{\exemple}[1]{%
  \fontspec{Amiri}[Scale=0.9,Script=Arabic]
  \RL{#1}
}
\newcommand{\annot}[1]{%
  \fontspec{Amiri}[Scale=0.7,Script=Arabic]
  \RL{#1}
}

\newcommand{\emphaseVV}[1]{\char"200D\textcolor{black!40}{\char"200D #1\char"200D}{\char"200D}}
\newcommand{\emphaseVF}[1]{\char"200D\textcolor{black!40}{\char"200D #1}}
\newcommand{\emphaseFV}[1]{\textcolor{black!40}{#1\char"200D}{\char"200D}}
\newcommand{\emphaseFF}{\textcolor{black!40}}

\def\pstreehooki{\psset{thislevelsep=*0pt}}
\def\pstreehookiii{\psset{thislevelsep=*0pt}}
\def\pstreehookv{\psset{thislevelsep=*0pt}}
\def\pstreehookvii{\psset{thislevelsep=*0pt}}
\def\pstreehookix{\psset{thislevelsep=*0pt}}
\def\pstreehookxi{\psset{thislevelsep=*0pt}}
\def\pstreehookxiii{\psset{thislevelsep=*0pt}}
\def\pstreehookxv{\psset{thislevelsep=*0pt}}

\pretitle{}\posttitle{}\title{}
\preauthor{}\postauthor{}
\author{Cours 7 - \RL{الجَوَازِمُ}}
\predate{}\postdate{}\date{}

\begin{document}
\begin{preview}
  \begin{center}
    
    \small %small,normalsize,large,Large,LARGE,huge,HUGE

    %pstree
    % \\ \begin{flushright}{\exemplefont  } \end{flushright}  \emphaseVF{}
    \pstree{\LFTwl{t}{\RL{الجَوَازِمُ\\مَا يُجْزِمُ فِعلاً واحِداً}}}{\pstree{\Tp[edge=none]}{%
    \LFTwl{t}{\RL{لا النَّاهِية \\[1\baselineskip]
        \exemplefont{لاَ تُشرِ\emphaseFF{كْ}
          بِاللّهِ\\ (مَجْرُور عَلى تَعظِيمِهِ)}
      }}
    \LFTwl{t}{\RL{لام الطَّلَبِية \\  [1\baselineskip]
    \exemplefont{مَن يُؤمِنُ بِاليَومِ الأَخِرِ فَلْيَقُ\emphaseVF{لْ}
      خَيراً أَوْ لِيَصْمُ\emphaseVF{تْ}}}}
    \LFTwl{t}{\RL{لَمْ / لَمَّا \\ حَرفُ جَزْمٍ و نَفِي وَ قَلبٍ \\[1\baselineskip]
        \exemplefont{\{\{لَم يَلِ\emphaseVF{دْ}
          وَ لَمْ يُولَ\emphaseVF{دْ}
          \}\}}}}
    \LFTw{t}{\RL{الطَّلَب \\[1\baselineskip]
        \exemplefont{اجتَهِدْ تَنجَ\emphaseVF{حْ}}}}}}

    \pstree{\LFTw{t}{\RL{  أدَوَاتُ الشَرط\\ مَا يُجْزِمُ فِعلَين}}}{\pstree{\Tp[edge=none]}{%
        \LFTwl{t}{\RL{أيُّ \\
            اِسمُ شَرط و جازِمٍ \\[1\baselineskip]
            \exemplefont{أَيُّ كِتابٍ تَقْرَ\emphaseFF{أْ}
              أَقرَ\emphaseFF{أْ}}}}
        \LFTwl{t}{\RL{مَتَى / أَيَّانَ \\
            اِسمُ شَرط و جازِمٍ لِلزَّمَان \\[1\baselineskip]
            \exemplefont{مَتَى/أَيَّانَ تَخرُ\emphaseFF{جْ}
              أَدْخُ\emphaseVF{لْ}}}}
        \LFTwl{t}{\RL{أَينَ / أنَّى / حَيثُيمَا \\
            اِسمُ شَرط و جازِمٍ لِلمَكَان \\[1\baselineskip]
            \exemplefont{أَينَ /أنَّى/حَيثُيمَا يَجلِ\emphaseVF{سْ}
              العَالِمُ يُحتَرَ\emphaseFF{مْ}}}}
        \LFTwl{t}{\RL{مَن \\
            اِسمُ شَرط و جازِمٍ لِلعَاقِلٍ \\[1\baselineskip]
            \exemplefont{\{\{وَمَنْ يَتَّ\emphaseVF{قِ}
              اللَّهَ يَجْعَ\emphaseVF{لْ}
              لَهُ مَخْرَجًا\}\}}}}
        \LFTwl{t}{\RL{مَا / مَهمَا \\
            اِسمُ شَرط و جازِمٍ لِغَيرِ عَاقِلٍ \\[1\baselineskip]
            \exemplefont{مَا/مَهمَا تَأكُ\emphaseVF{لْ}
              أَكُ\emphaseVF{لْ} \\
              \{\{وَمَا تَفْعَلُ\emphaseVF{وا}
              مِنْ خَيْرٍ يَعْلَ\emphaseVV{مْ}هُ
              اللَّـهُ\}\}}}}
        \LFTwl{t}{\RL{إِنْ / إِذمَا \\
            حَرف شَرط و جازِمٍ \\[1\baselineskip]
            \exemplefont{إِنْ/إِذمَا تَجْتَهِ\emphaseVF{دْ}
              تَنجَ\emphaseVF{حْ}}}}
      }}
  \end{center}
\end{preview}
\end{document}