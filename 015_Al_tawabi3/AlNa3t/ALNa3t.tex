% Generated by LaTreeX, <https:\\ \exemplefontlautgesetz.com/latreex/>
% TEX root = ../000_Complet/A4_008_6.tex
\documentclass{article}
\usepackage[a4paper,margin=1in,landscape]{geometry}
\usepackage[active,tightpage,xetex]{preview}
\usepackage[tuenc]{fontspec}
\usepackage{pstricks,pst-node,pst-tree}
\usepackage{varwidth,realscripts}
\usepackage{graphicx}
\usepackage{bidi}
\usepackage{titling}
\defaultfontfeatures{Mapping=tex-text}
\setmainfont{Amiri-Bold}[Script=Arabic,Scale=1.2]
\pagestyle{empty}
\psset{showbbox=false,
       treemode=D,
       linewidth=0.3pt,
       treesep=1em,
       levelsep=5em,
       arrows=->,
       edge=\ncangles,
       angleA=-90,
       angleB=90,
       armA=3em,
       armB=2em}
\newcommand{\LFTw}[2]{%
\Tr[name=#1,ref=t]{\psframebox[linestyle=none,framesep=3pt]{%
  \begin{varwidth}{10em}\center #2\end{varwidth}}}
}
\newcommand{\LFTwl}[2]{%
\Tr[ref=#1]{\psframebox[linestyle=none,framesep=3pt]{%
  \begin{varwidth}{18em}\center #2\end{varwidth}}}
}
\newcommand{\LFTwsolid}[2]{%
\Tr[ref=#1]{\psframebox[linestyle=solid,framesep=3pt]{%
  \begin{varwidth}{30ex}\center #2\end{varwidth}}}
}
\newcommand{\LFTr}[2]{%
\Tr[ref=#1]{%
  \psframebox[linestyle=none,framesep=2pt]{#2}}
}

\newfontfamily{\exemplefont}{Amiri}[Scale=1,Script=Arabic]
\newfontfamily{\moyennefont}{Amiri-Bold}[Scale=1,Script=Arabic]
\newcommand{\exemple}[1]{%
  \fontspec{Amiri}[Scale=1,Script=Arabic]
  \RL{#1}
}
\newcommand{\annot}[1]{%
  \fontspec{Amiri}[Scale=1,Script=Arabic]
  \RL{#1}
}

\newcommand{\emphaseVV}[1]{\char"200D\textcolor{black!40}{\char"200D #1\char"200D}{\char"200D}}
\newcommand{\emphaseVF}[1]{\char"200D\textcolor{black!40}{\char"200D #1}}
\newcommand{\emphaseFV}[1]{\textcolor{black!40}{#1\char"200D}{\char"200D}}
\newcommand{\emphaseFF}{\textcolor{black!40}}

\def\pstreehooki{\psset{thislevelsep=*0pt}}
\def\pstreehookiii{\psset{thislevelsep=*0pt}}
\def\pstreehookv{\psset{thislevelsep=*0pt}}
\def\pstreehookvii{\psset{thislevelsep=*0pt}}
\def\pstreehookix{\psset{thislevelsep=*0pt}}
\def\pstreehookxi{\psset{thislevelsep=*0pt}}
\def\pstreehookxiii{\psset{thislevelsep=*0pt}}
\def\pstreehookxv{\psset{thislevelsep=*0pt}}

\pretitle{}\posttitle{}\title{\RL{التَوَابع}}
\preauthor{}\postauthor{}
\author{}
\predate{}\postdate{}\date{}

\begin{document}
\begin{preview}
  \begin{center}

    \small %small,normalsize,large,Large,LARGE,huge,HUGE
    %pstree
    % \\ \exemplefont \begin{flushright}{\exemplefont  } \end{flushright}  \emphaseVV{}
    \pstree{\LFTw{t}{\RL{النّعت \\  \exemplefont التَّابِع المُشتَق أَو المؤَول بِهِ\\المُبَاين لِلَفظ مَتبُوعِهِ}}}{\pstree{\Tp[edge=none]}{%
      \pstree{\LFTw{t}{\RL{اِعرَاب النَّعت}}}{\pstree{\Tp[edge=none]}{%
        \LFTw{t}{\RL{سَبَبِي  \\يَتبَعُ \emphaseFF{الأَوَّال} فِي الإِغرَاب\\ وَ \underline{الثَّانِي} فِي التَّأنيث وَ التَذكِير \\ \exemplefont مَرَتُ \emphaseFF{بِرَجُلٍ} قَائِمَةٍ \underline{أُمُّهُ}\\مَرَرتُ \emphaseFF{بِإِمرَأَةٍ} قَائِمٍ \underline{أَبوهَا}}}
        \LFTw{t}{\RL{حَقيقي\\تَابِعَهُ فِي كُلِّ أَحكَامِه \\ \exemplefont جَاءَ زَيدٌ العَاقِلُ}}
      }}
      \pstree{\LFTw{t}{\RL{فَائِدَة النَّعت}}}{\pstree{\Tp[edge=none]}{%
        \LFTw{t}{\RL{التَّوكِيد \\ \exemplefont تِلكَ عشرَةٌ كَامِلَةٌ}}
        \LFTw{t}{\RL{التَّرَحُّم\\ \exemplefont  اللَّهُمَّ ارحَم عبدَكَ المِسكِينَ}}
        \LFTw{t}{\RL{الذَّم\\ \exemplefont أَعُوذُ بِاللَّه مِنَ الشَيطَانِ الرَّجيمِ}}
        \LFTw{t}{\RL{المَدح\\ \exemplefont بِسمِ اللَّهِ الرَّحمَنِ الرَّحيم}}
        \LFTw{t}{\RL{التَّوضيح\\ \exemplefont مَرَرتُ بِزَيدٍ الخيّاطِ}}
        \LFTw{t}{\RL{التَّخصِيص\\ \exemplefont مَرَرتُ بِرَجُلٍ كَاتِبٍ}}
      }}
    }}


  \end{center}
\end{preview}
\end{document}