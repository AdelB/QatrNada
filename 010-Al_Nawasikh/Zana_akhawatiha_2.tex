% Generated by LaTreeX, <https:\\ \exemplefontlautgesetz.com/latreex/>
% !TEX root = ../000_Complet/A4_010.tex
\documentclass{article}
\usepackage[a4paper,margin=1in,landscape]{geometry}
\usepackage[active,tightpage,xetex]{preview}
\usepackage[tuenc]{fontspec}
\usepackage{pstricks,pst-node,pst-tree}
\usepackage{varwidth,realscripts}
\usepackage{graphicx}
\usepackage{ulem}
\usepackage{bidi}
\usepackage{titling}
\defaultfontfeatures{Mapping=tex-text}
\setmainfont{Amiri-Bold}[Script=Arabic,Scale=1.2]
\pagestyle{empty}
\psset{showbbox=false,
       treemode=D,
       linewidth=0.3pt,
       treesep=1em,
       levelsep=4em,
       arrows=->,
       edge=\ncangles,
       angleA=-90,
       angleB=90,
       armA=2em,
       armB=2em}
\newcommand{\LFTw}[2]{%
\Tr[name=#1,ref=t]{\psframebox[linestyle=none,framesep=3pt]{%
  \begin{varwidth}{10em}\center #2\end{varwidth}}}
}
\newcommand{\LFTwl}[2]{%
\Tr[ref=#1]{\psframebox[linestyle=none,framesep=3pt]{%
  \begin{varwidth}{15em}\center #2\end{varwidth}}}
}
\newcommand{\LFTwsolid}[2]{%
\Tr[ref=#1]{\psframebox[linestyle=solid,framesep=3pt]{%
  \begin{varwidth}{30ex}\center #2\end{varwidth}}}
}
\newcommand{\LFTr}[2]{%
\Tr[ref=#1]{%
  \psframebox[linestyle=none,framesep=2pt]{#2}}
}

\newfontfamily{\exemplefont}{Amiri}[Scale=1,Script=Arabic]
\newfontfamily{\moyennefont}{Amiri-Bold}[Scale=1,Script=Arabic]
\newcommand{\exemple}[1]{%
  \fontspec{Amiri}[Scale=1,Script=Arabic]
  \RL{#1}
}
\newcommand{\annot}[1]{%
  \fontspec{Amiri}[Scale=1,Script=Arabic]
  \RL{#1}
}

\newcommand{\emphaseVV}[1]{\char"200D\textcolor{black!40}{\char"200D #1\char"200D}{\char"200D}}
\newcommand{\emphaseVF}[1]{\char"200D\textcolor{black!40}{\char"200D #1}}
\newcommand{\emphaseFV}[1]{\textcolor{black!40}{#1\char"200D}{\char"200D}}
\newcommand{\emphaseFF}{\textcolor{black!40}}

\def\pstreehooki{\psset{thislevelsep=*0pt}}
\def\pstreehookiii{\psset{thislevelsep=*0pt}}
\def\pstreehookv{\psset{thislevelsep=*0pt}}
\def\pstreehookvii{\psset{thislevelsep=*0pt}}
\def\pstreehookix{\psset{thislevelsep=*0pt}}
\def\pstreehookxi{\psset{thislevelsep=*0pt}}
\def\pstreehookxiii{\psset{thislevelsep=*0pt}}
\def\pstreehookxv{\psset{thislevelsep=*0pt}}

\pretitle{}\posttitle{}\title{\RL{نَاوَسِخ المُبتَدَأ وَ الخَبَر}}
\preauthor{}\postauthor{}
\author{}
\predate{}\postdate{}\date{}

\begin{document}
\begin{preview}
    \begin{center}
        
        \small %small,normalsize,large,Large,LARGE,huge,HUGE
        %pstree
        % \\ \exemplefont \begin{flushright}{\exemplefont } \end{flushright}  \emphaseVV{}
        \pstree{\LFTw{t}{\RL{أَحكَام ظَنَّ و أَخَوَاتُهَا  \\ (مَفعُول \emphaseFF{أَوَّل} وَ \underline{ثَانِي})}}}{\pstree{\Tp[edge=none]}{%
                \pstree{\LFTw{t}{\RL{التَّعلِيق \\ \moyennefont إِبطَال العَمَل لَفظاً لا مَحَلّاً \\ لِاعتِرَاض مَا لَهُ صَدرُ الكَلاَم :}}}{\pstree{\Tp[edge=none]}{%
                        \pstree{\LFTw{t}{\RL{الاستِفهَام}}}{\pstree{\Tp[edge=none]}{%
                                \LFTwl{t}{\RL{هَمزَةُ الاستِفهَام تَدخُل عَلى أَحَدَ المَفعُولَين \\ \exemplefont عَلِمتُ \emphaseFF{أَعَلِيُّ} مُسَافِرٌ \underline{أَم خَالِدٌ} \exemplefont }}
                                \LFTw{t}{\RL{أَحَدَ المَفعولَين اسم استِفهَام \\ \exemplefont عَلِمتُ \emphaseFF{أَيُّهُم} \underline{مُوَاظب} عَلى الحُضور \exemplefont }}
                            }}
                        \LFTw{t}{\RL{لاَم القَسَم \\ \exemplefont عَلِمتُ لَ\emphaseVF{يُحَاسَبَنَّ} \underline{المَرءُ} عَلى عَمَلِهِ \exemplefont }}
                        \LFTw{t}{\RL{لاَم الاِبتِدَاء \\ \exemplefont عَلِمتُ لَ\emphaseVF{لْبَلاَغَةُ} \underline{إِيجَازٌ} \exemplefont }}
                        \LFTw{t}{\RL{حُرُوف نَفي (مَا، لاَ، إِنْ) \\ \exemplefont عَلِمتُ مَا \emphaseFF{التَّحَيُّلُ} \underline{شَجَاعَةٌ}  \\ \exemplefont وَجَدتُ لاَ \emphaseFF{الإِفراطُ} \underline{مَحمُودٌ} وَ لاَ \underline{التَّفرِيطُ} \exemplefont }}
                    }}
                \pstree{\LFTw{t}{\RL{الإِلغَاء جَوَازاً\\ \moyennefont إِبطَال العَمَل لَفظاً و مَحَلّاً}}}{\pstree{\Tp[edge=none]}{%
                        \LFTw{t}{\RL{التَّأَخُر\\ \moyennefont إِلغاؤُهُ أَقوَى مِن إِعمَالِهِ  \\ \exemplefont \emphaseFF{زَيداً} \underline{عَالِماً} ظَنَنتُ   \\ \exemplefont \emphaseFF{زَيدٌ} \underline{عَالِمٌ} ظَنَنتُ \exemplefont }}
                        \LFTw{t}{\RL{التَّوَسُط \\ \moyennefont إِعمَالُهُ أَوَلى لِانَّهُ أَصل \\ \exemplefont \emphaseFF{زَيداً} ظَنَنتُ \underline{عَالِماً}  \\ \exemplefont \emphaseFF{زَيدٌ} ظَنَنتُ \underline{عَالِمٌ} \exemplefont }}
                    }}
                \LFTw{t}{\RL{الإِعمَال \\ \moyennefont نَصب المُبتَدَأ وَ الخَبَر}}
            }}
    \end{center}
\end{preview}
\end{document}