% Generated by LaTreeX, <https:\\ \exemplefontlautgesetz.com/latreex/>
% !TEX root = ../000_Complet/A4_010.tex
\documentclass{article}
\usepackage[a4paper,margin=1in,landscape]{geometry}
\usepackage[active,tightpage,xetex]{preview}
\usepackage[tuenc]{fontspec}
\usepackage{pstricks,pst-node,pst-tree}
\usepackage{varwidth,realscripts}
\usepackage{graphicx}
\usepackage{ulem}
\usepackage{bidi}
\usepackage{titling}
\defaultfontfeatures{Mapping=tex-text}
\setmainfont{Amiri-Bold}[Script=Arabic,Scale=1.2]
\pagestyle{empty}
\psset{showbbox=false,
       treemode=D,
       linewidth=0.3pt,
       treesep=1em,
       levelsep=4em,
       arrows=->,
       edge=\ncangles,
       angleA=-90,
       angleB=90,
       armA=2em,
       armB=2em}
\newcommand{\LFTw}[2]{%
\Tr[name=#1,ref=t]{\psframebox[linestyle=none,framesep=3pt]{%
  \begin{varwidth}{10em}\center #2\end{varwidth}}}
}
\newcommand{\LFTwl}[2]{%
\Tr[ref=#1]{\psframebox[linestyle=none,framesep=3pt]{%
  \begin{varwidth}{15em}\center #2\end{varwidth}}}
}
\newcommand{\LFTwsolid}[2]{%
\Tr[ref=#1]{\psframebox[linestyle=solid,framesep=3pt]{%
  \begin{varwidth}{30ex}\center #2\end{varwidth}}}
}
\newcommand{\LFTr}[2]{%
\Tr[ref=#1]{%
  \psframebox[linestyle=none,framesep=2pt]{#2}}
}

\newfontfamily{\exemplefont}{Amiri}[Scale=1,Script=Arabic]
\newfontfamily{\moyennefont}{Amiri-Bold}[Scale=1,Script=Arabic]
\newcommand{\exemple}[1]{%
  \fontspec{Amiri}[Scale=1,Script=Arabic]
  \RL{#1}
}
\newcommand{\annot}[1]{%
  \fontspec{Amiri}[Scale=1,Script=Arabic]
  \RL{#1}
}

\newcommand{\emphaseVV}[1]{\char"200D\textcolor{black!40}{\char"200D #1\char"200D}{\char"200D}}
\newcommand{\emphaseVF}[1]{\char"200D\textcolor{black!40}{\char"200D #1}}
\newcommand{\emphaseFV}[1]{\textcolor{black!40}{#1\char"200D}{\char"200D}}
\newcommand{\emphaseFF}{\textcolor{black!40}}


\pretitle{}\posttitle{}\title{\RL{نَاوَسِخ المُبتَدَأ وَ الخَبَر}}
\preauthor{}\postauthor{}
\author{}
\predate{}\postdate{}\date{}

\begin{document}
\begin{preview}
  \begin{center}
    
    \normalsize %small,normalsize,large,Large,LARGE,huge,HUGE
    %pstree
    % \\ \exemplefont \begin{flushright}{} \end{flushright}  \emphaseVV{}
        \pstree{\LFTw{t}{\RL{مَا تَختَص بِهِ (كَانَ)}}}{\pstree{\Tp[edge=none]}{%
        \LFTwl{t}{\RL{جَوَاز حَذفِهَا مَعَ اسمِهَا \\ \moyennefont بَعدَ (إِن) وَ (لَو) اشَّرطيَتَين \\  \exemplefont  المَرءُ مُحَاسَبٌ عَلَى عَمَلِهِ  إِن \underline{خَيراً } فَخَير وَ إن \underline{شَراً} فَشَر   \\  \exemplefont  التَمِس وَ لَو \underline{خَاتِماً} مِن حَدِيد  }}
        \LFTwl{t}{\RL{جَوَاز حَذفِهَا و إِبقَاء \emphaseFF{اسمِهَا} وَ \underline{خَبَرِهَا} \\ \moyennefont يَعُود عَنهَا (مَا) \\  \exemplefont  أبَا خُرَاشَة امَّا \emphaseFF{أَنتَ} \underline{ذَا نَفَرٍ}  \\  \exemplefont  أبَا خُرَاشَة انْ ما \emphaseFF{أَنتَ} \underline{ذَا نَفَرٍ}  \\  \exemplefont  أبَا خُرَاشَة  كُن\emphaseVF{تَ} \underline{ذَا نَفَرٍ}  }}
        \LFTwl{t}{\RL{جَوَاز زِيَادَتِهَا (تَوكِيد) \\ \moyennefont بَينَ شَيئَين مُتَلاَزِمَين \\  \exemplefont  الكِتَابُ كَانَ مُفِيدٌ  \\  \exemplefont  مَا كَانَ أَحسَنُ زَيداً  }}
        \LFTwl{t}{\RL{جَوَاز حَذف نونِهَا \\ \moyennefont الشُّرُوط \\ \begin{flushright} -مُضارِع:  {\exemplefont  أكُون}  \\ -مَجزُومَة بِالسُّكُن:  {\exemplefont  \emphaseFF{لَم} أَكُ\emphaseVF{نْ}}  \\  -لاَ مُتَّصل بِضَمير نَصب : {\exemplefont أَكُنهُ } \\ -لاَ مُتَّصل بِحَرف سَاكِن : {\exemplefont لَمْ يَكُ\emphaseVF{نِ ا}لَّذِينَ كَفَرُوا  }\\ -فِي حَالَةِ الوصل:  {\exemplefont  \emphaseFF{لَم} يَكُ \emphaseFF{طَالِبُ} العِلم مقصِّراً .} \end{flushright}}}
      }}
  \end{center}
\end{preview}
\end{document}