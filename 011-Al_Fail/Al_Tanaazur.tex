% Generated by LaTreeX, <https:\\ \exemplefontlautgesetz.com/latreex/>
% TEX root = ../000_Complet/A4_011.tex
\documentclass{article}
\usepackage[a4paper,margin=1in,landscape]{geometry}
\usepackage[active,tightpage,xetex]{preview}
\usepackage[tuenc]{fontspec}
\usepackage{pstricks,pst-node,pst-tree}
\usepackage{varwidth,realscripts}
\usepackage{graphicx}
\usepackage{ulem}
\usepackage{bidi}
\usepackage{titling}

\defaultfontfeatures{Mapping=tex-text}
\setmainfont{Amiri-Bold}[Script=Arabic,Scale=1.2]
\pagestyle{empty}
\psset{showbbox=false,
       treemode=D,
       linewidth=0.3pt,
       treesep=1em,
       levelsep=2em,
       arrows=->,
       edge=\ncangles,
       angleA=-90,
       angleB=90,
       armA=1em,
       armB=1em}
\newcommand{\LFTw}[2]{%
\Tr[name=#1,ref=t]{\psframebox[linestyle=none,framesep=3pt]{%
  \begin{varwidth}{10em}\center #2\end{varwidth}}}
}
\newcommand{\LFTwl}[2]{%
\Tr[ref=#1]{\psframebox[linestyle=none,framesep=3pt]{%
  \begin{varwidth}{20em}\center #2\end{varwidth}}}
}
\newcommand{\LFTwsolid}[2]{%
\Tr[ref=#1]{\psframebox[linestyle=solid,framesep=3pt]{%
  \begin{varwidth}{30ex}\center #2\end{varwidth}}}
}
\newcommand{\LFTr}[2]{%
\Tr[ref=#1]{%
  \psframebox[linestyle=none,framesep=2pt]{#2}}
}

\newfontfamily{\exemplefont}{Amiri}[Scale=1,Script=Arabic]
\newfontfamily{\moyennefont}{Amiri-Bold}[Scale=1,Script=Arabic]
\newcommand{\exemple}[1]{%
  \fontspec{Amiri}[Scale=1,Script=Arabic]
  \RL{#1}
}
\newcommand{\annot}[1]{%
  \fontspec{Amiri}[Scale=1,Script=Arabic]
  \RL{#1}
}

\newcommand{\emphaseVV}[1]{\char"200D\textcolor{black!40}{\char"200D #1\char"200D}{\char"200D}}
\newcommand{\emphaseVF}[1]{\char"200D\textcolor{black!40}{\char"200D #1}}
\newcommand{\emphaseFV}[1]{\textcolor{black!40}{#1\char"200D}{\char"200D}}
\newcommand{\emphaseFF}{\textcolor{black!40}}


\newcommand{\ulineVF}[1]{\char"200D\underline{\char"200D #1}}
\newcommand{\soutVF}[1]{\char"200D\sout{\char"200D #1}}


\def\pstreehooki{\psset{thislevelsep=*0pt}}
\def\pstreehookiii{\psset{thislevelsep=*0pt}}
\def\pstreehookv{\psset{thislevelsep=*0pt}}
\def\pstreehookvii{\psset{thislevelsep=*0pt}}
\def\pstreehookix{\psset{thislevelsep=*0pt}}
\def\pstreehookxi{\psset{thislevelsep=*0pt}}
\def\pstreehookxiii{\psset{thislevelsep=*0pt}}
\def\pstreehookxv{\psset{thislevelsep=*0pt}}

\begin{document}
\begin{preview}
  \begin{center}
    \small %small,normalsize,large,Large,LARGE,huge,HUGE
    %pstree
    % \\ \exemplefont \begin{flushright}{\exemplefont} \end{flushright}  \emphaseVV{}
    \pstree{\LFTwl{t}{\RL{التَّنَازُع - الإِعمَال \\ أَن يَتَقَدَّمَ \underline{عَامِلاَن أَو أَكثَر} وَ يَتَأَخَّر \emphaseFF{مَعمُول أَو أَكثَر}\\ وَ يَكُون كُلٌّ مِنَ المُتَقَدِّم طَالِباً لِذَلِكَ المُتَأَخِّر}}}{\pstree{\Tp[edge=none]}{%
        \pstree{\LFTw{t}{\RL{مَا لَيسَ مِن بَابِ التَنَازُع }}}{\pstree{\Tp[edge=none]}{%
            \pstree{\LFTw{t}{\RL{\exemplefont وَ لَو أَن مَا أَسعَى لِأَدنَى مَعيشَة كَفَانِي وَ لَم أَطلُب قَليلٌ مِنَ المَال  \\ }}}{\pstree{\Tp[edge=none]}{%
                \LFTw{t}{\RL{\exemplefont وَ لَو أَن مَا أَسعَى لِأَدنَى مَعيشَة كَفَانِي  قَليلٌ مِنَ المَال وَ لَم أَطلُبِ امُلك  }}
              }}
          }}
        \pstree{\LFTw{t}{\RL{الخِلاَف فِي الاِختِيَار}}}{\pstree{\Tp[edge=none]}{%
            \LFTwl{t}{\RL{الأخِير لِقُربِه (البَصرِيون) \\ \moyennefont إِنِ احتَاجَ الأَوَّل إِلى مَرفُوع \underline{\underline{أَضمَرتَهُ}}\\وَ إِنِ احتَاجَ إِلى مَنصُوب أَو مَجرُور \sout{حَذفتَه} \\ \exemplefont \underline{قَامَ\ulineVF{ا}} وَ \underline{قَعَدَ} \emphaseFF{أَخَوَاكَ}  \\ \exemplefont ضَرَبَتُ\soutVF{هُمَا} وَ ضَرَبَنِي \emphaseFF{أَخَوَاكَ}  \\ \exemplefont مَرَرتُ \soutVF{بِهِمَا} وَ \underline{مَرَّ} بِي \emphaseFF{أَخَوَاك } \\}}
            \LFTwl{t}{\RL{الأوَّل لِسَبْقِه (الكُوفِيون) \\ \moyennefont \underline{\underline{أَضمَرتَ}} فِي الثَّانِي كُلّ مَا يَحتَاج إِلَيهِ \\مِن مَرفُوع وَ مَنصُوب وَ مَجرُور \\ \exemplefont \underline{قَامَ} وَ \underline{قَعَدَ\underline{ا}} \emphaseFF{أَخواكَ}   \\ \exemplefont \underline{قَامَ} وَ \underline{ضَرَبتُ\ulineVF{هُمَا}} \emphaseFF{أَخواكَ}   \\ \exemplefont \underline{قَامَ} وَ \underline{مَرَرتُ بِ\ulineVF{هِمَا}} \emphaseFF{أَخواكَ}   }}
          }}
        \pstree{\LFTw{t}{\RL{مِثَال التَّنَازُع}}}{\pstree[thistreesep=0.5em]{\Tp[edge=none]}{%
            \LFTwl{t}{\RL{ أَكثَر مِن عَامِلَين \\أَكثَر مِن مَعمُول \\ \exemplefont \underline{تُسَبِحُونَ} وَ \underline{تُحَمِّدُونَ} وَ \underline{تُكَبِّرُونَ} \\ \emphaseFF{دُبُرَ كُلِّ صَلاَةٍ} \emphaseFF{ثَلاَثاً وَ ثَلاَثٍ}  }}
            \LFTwl{t}{\RL{ أَكثَر مِن عَامِلَين \\مَعمُولاً وَاحِداً \\ \exemplefont كَمَا  \underline{صَلَيتَ} وَ \underline{بَارَكتَ}\\ وَ\underline{تَرَحَمتَ} \emphaseFF{عَلى إِبرَاهِم}  }}
            \LFTwl{t}{\RL{ العَامِلَين \\أَكثَر مِن مَعمُول \\ \exemplefont \underline{ضَرَبَ} وَ \underline{أَكرَمَ} \emphaseFF{زَيدٌ} \emphaseFF{عَمراً} }}
            \LFTwl{t}{\RL{ العَامِلَين \\ مَعمُولاً وَاحِداً \\ \exemplefont \underline{أَتُونِي} \underline{أُفرِغ} \emphaseFF{عَلَيهِ قِطراً}  }}
          }}
      }}
  \end{center}
\end{preview}
\end{document}