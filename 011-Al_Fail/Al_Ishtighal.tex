% Generated by LaTreeX, <https:\\ \exemplefontlautgesetz.com/latreex/>
% TEX root = ../000_Complet/A4_011.tex
\documentclass{article}
\usepackage[a4paper,margin=1in,landscape]{geometry}
\usepackage[active,tightpage,xetex]{preview}
\usepackage[tuenc]{fontspec}
\usepackage{pstricks,pst-node,pst-tree}
\usepackage{varwidth,realscripts}
\usepackage{graphicx}
\usepackage{ulem}
\usepackage{bidi}
\usepackage{titling}

\defaultfontfeatures{Mapping=tex-text}
\setmainfont{Amiri-Bold}[Script=Arabic,Scale=1.2]
\pagestyle{empty}
\psset{showbbox=false,
       treemode=D,
       linewidth=0.3pt,
       treesep=1em,
       levelsep=2em,
       arrows=->,
       edge=\ncangles,
       angleA=-90,
       angleB=90,
       armA=1em,
       armB=1em}
\newcommand{\LFTw}[2]{%
\Tr[name=#1,ref=t]{\psframebox[linestyle=none,framesep=3pt]{%
  \begin{varwidth}{10em}\center #2\end{varwidth}}}
}
\newcommand{\LFTwl}[2]{%
\Tr[ref=#1]{\psframebox[linestyle=none,framesep=3pt]{%
  \begin{varwidth}{20em}\center #2\end{varwidth}}}
}
\newcommand{\LFTwsolid}[2]{%
\Tr[ref=#1]{\psframebox[linestyle=solid,framesep=3pt]{%
  \begin{varwidth}{30ex}\center #2\end{varwidth}}}
}
\newcommand{\LFTr}[2]{%
\Tr[ref=#1]{%
  \psframebox[linestyle=none,framesep=2pt]{#2}}
}

\newfontfamily{\exemplefont}{Amiri}[Scale=1,Script=Arabic]
\newfontfamily{\moyennefont}{Amiri-Bold}[Scale=1,Script=Arabic]
\newcommand{\exemple}[1]{%
  \fontspec{Amiri}[Scale=1,Script=Arabic]
  \RL{#1}
}
\newcommand{\annot}[1]{%
  \fontspec{Amiri}[Scale=1,Script=Arabic]
  \RL{#1}
}

\newcommand{\emphaseVV}[1]{\char"200D\textcolor{black!40}{\char"200D #1\char"200D}{\char"200D}}
\newcommand{\emphaseVF}[1]{\char"200D\textcolor{black!40}{\char"200D #1}}
\newcommand{\emphaseFV}[1]{\textcolor{black!40}{#1\char"200D}{\char"200D}}
\newcommand{\emphaseFF}{\textcolor{black!40}}


\newcommand{\ulineVF}[1]{\char"200D\underline{\char"200D #1}}
\newcommand{\soutVF}[1]{\char"200D\sout{\char"200D #1}}


\def\pstreehooki{\psset{thislevelsep=*0pt}}
\def\pstreehookiii{\psset{thislevelsep=*0pt}}
\def\pstreehookv{\psset{thislevelsep=*0pt}}
\def\pstreehookvii{\psset{thislevelsep=*0pt}}
\def\pstreehookix{\psset{thislevelsep=*0pt}}
\def\pstreehookxi{\psset{thislevelsep=*0pt}}
\def\pstreehookxiii{\psset{thislevelsep=*0pt}}
\def\pstreehookxv{\psset{thislevelsep=*0pt}}

\begin{document}
\begin{preview}
  \begin{center}
    \small %small,normalsize,large,Large,LARGE,huge,HUGE
    %pstree
    % \\ \exemplefont \begin{flushright}{\exemplefont} \end{flushright}  \emphaseVV{}
    \pstree{\LFTwl{t}{\RL{الاِشتِغَال \\ \moyennefont أَن يَتَقَدَّمَ الاِسم وَ يَتَأَخَر عَنهُ الَفِعل عَامِل فِي ضَمِير\\ وَ يَكُون ذَلِكَ الفِعل بِحَيث لَو فَرَغَ مِن ذَلِكَ المَعمُول \\ وَ سُلِّطَ عَلى الاِسم الأَوَّل \emphaseFF{نَصَبَهُ}}}}{\pstree{\Tp[edge=none]}{%
        \LFTw{t}{\RL{ \exemplefont  \emphaseFF{زَيداً} مَرَرتُ بِهِ \\ جَوَازتُ زَيداً مَرَتُ بِهِ}}
        \LFTw{t}{\RL{ \exemplefont  \emphaseFF{زَيداً} ضَرَبتُ أَخَاه\\ أَهنتُ زَيداً ضَرَبتُ أَخَاه}}
        \LFTw{t}{\RL{ \exemplefont  \emphaseFF{زَيداً} ضَرَبتُهُ \\ ضَرَبتُ زَيداً ضَرَبتُهُ}}
      }}
      
    \begin{center}\line(1,0){450}\end{center}

    \pstree{\LFTwl{t}{\RL{أَحوال الاِسم المُتَقَدِّم \\ \moyennefont  إِن كَانَ الاِسم مَرفُوعاً فَهوَ مُبتَدَأ وَ إِن كَانَ مَنصُوباً فَهُوَ مَفعُول بِهِ مُتَقَدَّم}}}{\pstree{\Tp[edge=none]}{%
    \LFTw{t}{\RL{يستَوي فِيه الأَمرَان \\ \exemplefont  زَيدٌ قَامَ أَبُوهُ وَ عَمراً أَكرَمتُهُ \\ \exemplefont  زَيدٌ قَامَ أَبُوهُ وَ عَمرٌ أَكرَمتُهُ}}
    \pstree{\LFTw{t}{\RL{وجُوب الرَفع \\ أدَاة خَاصَّة بِالاِسم}}}{\pstree{\Tp[edge=none]}{%
        \LFTw{t}{\RL{"إِذَا" الفُجَائِية \\ \exemplefont  خَرَجتُ فَإِذا زَيدٌ يَضرِبُهُ عَمرُو}}
      }}
    \pstree{\LFTw{t}{\RL{وجُوب النَصب \\ \moyennefont إِذَا تَقَدَّمَ عَلى الاِسم أَدَاة خَاصّة بَِلفِعل \\ أمثَال}}}{\pstree{\Tp[edge=none]}{%
        \LFTw{t}{\RL{أَدَوَات التَحضيض\\ \exemplefont  هَلاَّ زَيداً أَكرَمتُهُ}}
        \LFTw{t}{\RL{أَدَوَات الشَرط \\  \exemplefont  إِن زَيداً رَأَيتَهُ فَأَكرِمه}}
      }}
    \pstree{\LFTw{t}{\RL{تَرجِيح النَصب}}}{\pstree{\Tp[edge=none]}{%
    \LFTw{t}{\RL{أَن يَكُونَ الاِسم مُقتَرِناً بِعاطِف مَسبوق بِجُملَة فِعلية \\  \exemplefont  قَامَ زَيدٌ \emphaseFF{وَ} عَمراً أَكرَمتُهُ}}
    \LFTw{t}{\RL{أَن يَتَقَدَّمَ عَلى الاِسم أَدَاة الغَالِب عَلَيهَا أَن تَدخُلَ عَلى الأَفعال \\  \exemplefont  \emphaseFF{أَ} زَيداً ضَرَبتُه}}
    \LFTw{t}{\RL{الطَّلَبَ \\ \moyennefont الجمُلَة الطَّلَبِيَّة لاَ تَحتَمِل\\الصِدق وَ الكَذِب \\ الأَمر:  {\exemplefont  زَيداً اِضرِبه} \\ النَهي:  {\exemplefont  زَيداً لاَ تَهِنه}\\ الدُّعَاء:  {\exemplefont  اللّهُمَ عَبدَكَ ارَحَمه}}}
    }}
    \LFTw{t}{\RL{تَرجِيح الرَفع \\  \exemplefont  زَيدٌ ضَرَبتُهُ}}
    }}
  \end{center}
\end{preview}
\end{document}