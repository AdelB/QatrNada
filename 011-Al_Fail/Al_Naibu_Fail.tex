% Generated by LaTreeX, <https:\\ \exemplefontlautgesetz.com/latreex/>
% !TEX root = ../000_Complet/A4_011.tex
\documentclass{article}
\usepackage[a4paper,margin=1in,landscape]{geometry}
\usepackage[active,tightpage,xetex]{preview}
\usepackage[tuenc]{fontspec}
\usepackage{pstricks,pst-node,pst-tree}
\usepackage{varwidth,realscripts}
\usepackage{graphicx}
\usepackage{ulem}
\usepackage{bidi}
\usepackage{titling}

\defaultfontfeatures{Mapping=tex-text}
\setmainfont{Amiri-Bold}[Script=Arabic,Scale=1.2]
\pagestyle{empty}
\psset{showbbox=false,
       treemode=D,
       linewidth=0.3pt,
       treesep=0.5em,
       levelsep=2em,
       arrows=->,
       edge=\ncangles,
       angleA=-90,
       angleB=90,
       armA=1em,
       armB=1em}
\newcommand{\LFTw}[2]{%
\Tr[name=#1,ref=t]{\psframebox[linestyle=none,framesep=3pt]{%
  \begin{varwidth}{10em}\center #2\end{varwidth}}}
}
\newcommand{\LFTwl}[2]{%
\Tr[ref=#1]{\psframebox[linestyle=none,framesep=3pt]{%
  \begin{varwidth}{15em}\center #2\end{varwidth}}}
}
\newcommand{\LFTwsolid}[2]{%
\Tr[ref=#1]{\psframebox[linestyle=solid,framesep=3pt]{%
  \begin{varwidth}{30ex}\center #2\end{varwidth}}}
}
\newcommand{\LFTr}[2]{%
\Tr[ref=#1]{%
  \psframebox[linestyle=none,framesep=2pt]{#2}}
}

\newfontfamily{\exemplefont}{Amiri}[Scale=1,Script=Arabic]
\newfontfamily{\moyennefont}{Amiri-Bold}[Scale=1,Script=Arabic]
\newcommand{\exemple}[1]{%
  \fontspec{Amiri}[Scale=1,Script=Arabic]
  \RL{#1}
}
\newcommand{\annot}[1]{%
  \fontspec{Amiri}[Scale=1,Script=Arabic]
  \RL{#1}
}

\newcommand{\emphaseVV}[1]{\char"200D\textcolor{black!40}{\char"200D #1\char"200D}{\char"200D}}
\newcommand{\emphaseVF}[1]{\char"200D\textcolor{black!40}{\char"200D #1}}
\newcommand{\emphaseFV}[1]{\textcolor{black!40}{#1\char"200D}{\char"200D}}
\newcommand{\emphaseFF}{\textcolor{black!40}}

\def\pstreehooki{\psset{thislevelsep=*0pt}}
\def\pstreehookiii{\psset{thislevelsep=*0pt}}
\def\pstreehookv{\psset{thislevelsep=*0pt}}
\def\pstreehookvii{\psset{thislevelsep=*0pt}}
\def\pstreehookix{\psset{thislevelsep=*0pt}}
\def\pstreehookxi{\psset{thislevelsep=*0pt}}
\def\pstreehookxiii{\psset{thislevelsep=*0pt}}
\def\pstreehookxv{\psset{thislevelsep=*0pt}}

\begin{document}
\begin{preview}
  \begin{center}
    \small %small,normalsize,large,Large,LARGE,huge,HUGE
    %pstree
    % \\ \exemplefont \begin{flushright}{\exemplefont} \end{flushright}  \emphaseVV{}
    \pstree{\LFTw{t}{\RL{النَائِبُ عَنِ الفَاعِل \\ مَا أُقِيمَ مُقَامُ الفَاعِل بَعدَ حَذفِه}}}{\pstree{\Tp[edge=none]}{%
        \pstree{\LFTw{t}{\RL{تَعِيير الفِعل}}}{\pstree{\Tp[edge=none]}{%
            \LFTw{t}{\RL{مُضارِع \\ \moyennefont ضَم أَوله و \\فتح مَا قَبلَ آخِره\\ \exemplefont يَضرِبُ: يُضرَبُ  }}
            \pstree{\LFTw{t}{\RL{مَاضي \\ \moyennefont ضَم أَوله و \\كسر مَا قَبلَ آخِره}}}{\pstree{\Tp[edge=none]}{%
                \LFTw{t}{\exemplefont \begin{tabular}{rrr}
                  اُنفُعِلَ & $\leftarrow$ & انفَعَلَ  \\
                  أُفتُعِلَ & $\leftarrow$ &اِفتَعَلَ \\
                  اُفعُّلَ & $\leftarrow$ &اِفعَلَّ\\
                  اُفعُنلَل& $\leftarrow$ &اِفعَنلَلَ \\
                  اُفعُلَلَّ & $\leftarrow$ &اِفعَلَلَّ \\
                  اُستُفعِل& $\leftarrow$ &اِستِفعَلَ 
                   \end{tabular}}
                \LFTw{t}{\RL{مُعتَلّ العَيِن\\ \exemplefont قَالَ : قِيلَ }}
                \LFTw{t}{\RL{الأَصل\\ \exemplefont ضَرَبَ : ضُرِبَ  }}
              }}
          }}
        \pstree{\LFTw{t}{\RL{مَا يَنُوبُ عَنِ الفَاعِل}}}{\pstree{\Tp[edge=none]}{%
            \LFTw{t}{\RL{الجَر و المَجرُور \\ \exemplefont مُرَّ بِزَيدٍ}}
            \LFTw{t}{\RL{المَصدَر \\ \exemplefont جَلَستُ جُلُوس الأَمِير}}
            \pstree{\LFTw{t}{\RL{الظَرف \\ \exemplefont صِيمَ رَمَضَانُ}}}{\pstree{\Tp[edge=none]}{%
                \LFTw{t}{\RL{أَن لاَ يَكُون \\ المفَعُول بِهِ مَوجُوداً}}
                \LFTw{t}{\RL{مُتصَرِّف \\ \exemplefont جُلِسَ عِندَكَ}}
                \LFTw{t}{\RL{مُختَص \\ \exemplefont سِيرَ زَمَنٌ طَويلٌ \\ \exemplefont سِيرَ زَمَنٌ  }}
              }}
            \LFTw{t}{\RL{المَفعُول بِهِ \\ \moyennefont (الصل فِي النِيَابَة)\\ \exemplefont ضُرِبَ عَمرٌ  }}
          }}
        \pstree{\LFTw{t}{\RL{يَجُوز حَذف الفَاعِل إِمَّا}}}{\pstree{\Tp[edge=none]}{%
            \LFTw{t}{\RL{لِغَرَض مَعنَوي\\(كَالعِلمِ بِه) \\ \exemplefont ﴿إِذَا قِيلَ لَكُمْ تَفَسَّحُوا فِي الْمَجَالِسِ فَافْسَحُوا﴾  }}
            \LFTw{t}{\RL{لِغَرَض لَفظِي\\(كَتَخفِيف الجُملَة)\\ \exemplefont مَن طَابَت سَريرَتُهُ   \\ \exemplefont حُمِدَتْ سِيرَتُهُ  }}
            \LFTw{t}{\RL{الجَهل بِه \\ \exemplefont سُرِقَ المَتَاعُ  \\ \exemplefont رُوِيَ عَن رَسُولِ اللّه ﷺ }}
          }}
      }}
  \end{center}
\end{preview}
\end{document}