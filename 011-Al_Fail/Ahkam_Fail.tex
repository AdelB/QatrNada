% Generated by LaTreeX, <https:\\ \exemplefontlautgesetz.com/latreex/>
% !TEX root = ../000_Complet/A4_011.tex
\documentclass{article}
\usepackage[a3paper,margin=1in,landscape]{geometry}
\usepackage[active,tightpage,xetex]{preview}
\usepackage[tuenc]{fontspec}
\usepackage{pstricks,pst-node,pst-tree}
\usepackage{varwidth,realscripts}
\usepackage{graphicx}
\usepackage{ulem}
\usepackage{bidi}
\usepackage{titling}
\defaultfontfeatures{Mapping=tex-text}
\setmainfont{Amiri-Bold}[Script=Arabic,Scale=1.2]
\pagestyle{empty}
\psset{showbbox=false,
       treemode=D,
       linewidth=0.3pt,
       treesep=1em,
       levelsep=4em,
       arrows=->,
       edge=\ncangles,
       angleA=-90,
       angleB=90,
       armA=2em,
       armB=2em}
\newcommand{\LFTw}[2]{%
\Tr[name=#1,ref=t]{\psframebox[linestyle=none,framesep=3pt]{%
  \begin{varwidth}{10em}\center #2\end{varwidth}}}
}
\newcommand{\LFTwl}[2]{%
\Tr[ref=#1]{\psframebox[linestyle=none,framesep=3pt]{%
  \begin{varwidth}{15em}\center #2\end{varwidth}}}
}
\newcommand{\LFTwsolid}[2]{%
\Tr[ref=#1]{\psframebox[linestyle=solid,framesep=3pt]{%
  \begin{varwidth}{30ex}\center #2\end{varwidth}}}
}
\newcommand{\LFTr}[2]{%
\Tr[ref=#1]{%
  \psframebox[linestyle=none,framesep=2pt]{#2}}
}

\newfontfamily{\exemplefont}{Amiri}[Scale=1,Script=Arabic]
\newfontfamily{\moyennefont}{Amiri-Bold}[Scale=1,Script=Arabic]
\newcommand{\exemple}[1]{%
  \fontspec{Amiri}[Scale=1,Script=Arabic]
  \RL{#1}
}
\newcommand{\annot}[1]{%
  \fontspec{Amiri}[Scale=1,Script=Arabic]
  \RL{#1}
}

\newcommand{\emphaseVV}[1]{\char"200D\textcolor{black!40}{\char"200D #1\char"200D}{\char"200D}}
\newcommand{\emphaseVF}[1]{\char"200D\textcolor{black!40}{\char"200D #1}}
\newcommand{\emphaseFV}[1]{\textcolor{black!40}{#1\char"200D}{\char"200D}}
\newcommand{\emphaseFF}{\textcolor{black!40}}

\def\pstreehooki{\psset{thislevelsep=*0pt}}
\def\pstreehookiii{\psset{thislevelsep=*0pt}}
\def\pstreehookv{\psset{thislevelsep=5em,armA=3em,armB=1em}}
\def\pstreehookvii{\psset{thislevelsep=*0pt}}
\def\pstreehookix{\psset{thislevelsep=*0pt}}
\def\pstreehookxi{\psset{thislevelsep=*0pt}}
\def\pstreehookxiii{\psset{thislevelsep=*0pt}}
\def\pstreehookxv{\psset{thislevelsep=*0pt}}

\pretitle{}\posttitle{}\title{\RL{الفَاعِل}}
\preauthor{}\postauthor{}
\author{}
\predate{}\postdate{}\date{}

\begin{document}
\begin{preview}
  \begin{center}
    
    \normalsize %small,normalsize,large,Large,LARGE,huge,HUGE
    %pstree
    % \\ \exemplefont \begin{flushright}{\exemplefont} \end{flushright}  \emphaseVV{}
    \pstree{\LFTw{t}{\RL{أَحكَام \emphaseFF{الفَاعِل}}}}{\pstree{\Tp[edge=none]}{%
    \pstree{\LFTw{t}{\RL{يَجُوز حَذف الفَاعِل}}}{\pstree{\Tp[edge=none]}{%
        \LFTw{t}{\RL{فَاعِل المَصدَر \\ \exemplefont ﴿أَوْ \underline{إِطْعَامٌ} فِي يَوْمٍ ذِي مَسْغَبَةٍ﴿١٤﴾يَتِيمًا﴾}}
        \LFTw{t}{\RL{قَبلَ (إِلاَّ) \\ \exemplefont مَا \underline{قَامَ} إِلاَّ هِندٌ}}
        \LFTw{t}{\RL{الجَهل بِهِ \\ \exemplefont \underline{سُرِقَ} المَتَاعُ \\ \exemplefont \underline{رُوِيَ} عَنِ النَّبِي ﷺ  }}
        \LFTw{t}{\RL{فِعل عَلى وَزن(أَفعِلْ) التَعَجُّب فِي صِيغَةِ الأَمر\\ \exemplefont ﴿\underline{أَسمِعْ} \emphaseFF{بِهِم} وَ \underline{أَبصِرْ}﴾ \exemplefont (\emphaseFF{بِهِم}) \\ \exemplefont مَا أَسمَع هَؤُلاَءِ الكَافِرين وَ مَا أَبصَرَهُم فِي ذَلِكَ اليَوم }}
      }}
    \pstree{\LFTw{t}{\RL{تَأَخُّر الفاعِل عَنِ المَفعُول}}}{\pstree{\Tp[edge=none]}{%
        \pstree{\LFTw{t}{\RL{وجُوباً}}}{\pstree{\Tp[edge=none]}{%
            \LFTw{t}{\RL{أَن يَكُون المفعُول ضَمِيراً مُتَّصِلاً بِالفِعل \\ \exemplefont \underline{رَآ}نِي \emphaseFF{صَالِحٌ}  \\ \exemplefont \sout{  \underline{رَأَى} \emphaseFF{صَالِحٌ} إِياي  }}}
            \LFTw{t}{\RL{أَن يَتَّصِل بِالفَاعِل ضَمِير المَفعُول \\ \moyennefont لاَ يَجُوز أَن عَادَ الضَّمِير عَلى مُتَأَخِر لَفظاً وَ رُتبَةً ، إِلاَّ فِي ضَرُورة \\ \exemplefont ﴿وَإِذِ \underline{ابْتَلَىٰ} إِبْرَاهِيمَ \emphaseFF{رَبُّهُ}﴾  }}
            \LFTw{t}{\RL{أَن يَحصُرَ المفعُول فِي الفاعِل بِ(إِنَّمَا) أَو (إِلَّا) \\ \exemplefont ﴿إِنَّمَا \underline{يَخْشَى} اللَّـهَ مِنْ عِبَادِهِ \emphaseFF{الْعُلَمَاءُ}﴾  }}
          }}
        \LFTw{t}{\RL{جَوَازاً \\ \moyennefont لِأَنَّهُ مُتَأَخِّر لَفظاً مُتَقدِّم رُتبَةً \\ \exemplefont ﴿وَلَقَدْ \underline{جَاءَ} آلَ فِرْعَوْنَ \emphaseFF{النُّذُرُ}﴾  \\ \exemplefont كَمَا \underline{أتَى} رَبَّهُ \emphaseFF{مُوسَى} عَلى قَدَرِ  }}
      }}
    \pstree{\LFTw{t}{\RL{إِذَا كَانَ مُؤَنَّثاً لَحِقَ \underline{عَامِلُهُ} تَاءُ التَّأنِيث السَّاكِنة}}}{\pstree{\Tp[edge=none]}{%
        \LFTw{t}{\RL{فِعل مَاضٍ \\ \exemplefont \underline{قَامَتْ} \emphaseFF{هِندٌ}  }}
        \LFTw{t}{\RL{وَصف \\ \exemplefont زَيدٌ \underline{قَائِمَةٌ} \emphaseFF{أُمُّهُ}  }}
      }}
    \pstree{\LFTwl{t}{\RL{لاَ يَلحَق \underline{عَامِلُهُ} عَلاَمَة تَثنِيَة وَ لاَ جَمَع \\ \exemplefont \underline{قَامَ} \emphaseFF{زَيدَانِ}\\ \underline{قَامَ} \emphaseFF{زَيدونَ}}}}{\pstree{\Tp[edge=none]}{%
    \pstree{\LFTw{t}{\RL{إِلاَّ فِي لُغَة(أكَلونِي البَرَاغِيث) \\ \exemplefont يتعاقبون فيكم ملاَئِكة [...] \\}}}{\pstree{\Tp[edge=none]}{%
      \LFTwl{t}{\RL{ \moyennefont الوَاوُ دَالَ عَلى الجَمع وَ البَرَاغِيث فَاعِل}}
      \LFTwl{t}{\RL{ \moyennefont الوَاوُ فَاعِل وَ البَرَاغِيث بَدَل}}
    }}
    }}
    \LFTw{t}{\RL{أَن لاَ يَتَأَخَّر عَامِلَهُ عَنِهُ\\ \exemplefont قَامَ \emphaseFF{أَخَوَاكَ} \\  \sout{أَخَوَاكَ قَامَ}}}
    }}

    \begin{center}\line(1,0){450}\end{center}

    \pstree{\LFTw{t}{\RL{أَحكَام تَاء التَّأنِيث السَّاكِنة}}}{\pstree{\Tp[edge=none]}{%
        \pstree{\LFTw{t}{\RL{وَاجِب}}}{\pstree{\Tp[edge=none]}{%
            \LFTw{t}{\RL{أَن يَكُون الفِاعل ضَمِيراً مُتَّصِلاً \\ \exemplefont الشَّمس (\emphaseFF{هِيَ}) \underline{طَلَعَت}  }}
            \LFTw{t}{\RL{المُؤَنَّث اسم ظَاهِر حَقِيقِي غَيرِ مُنفَصِل\\ \exemplefont ﴿إِذْ \underline{قَالَتِ} \emphaseFF{امْرَأَةُ} عِمْرَانَ﴾ \\ \exemplefont \sout{\underline{قَالَ} \emphaseFF{امْرَأَةُ} عِمْرَانَ}}}
          }}
        \pstree{\LFTw{t}{\RL{جَوَازاً}}}{\pstree{\Tp[edge=none]}{%
            \LFTw{t}{\RL{الفَاعِل جَمع تَكسِير \\ \exemplefont \underline{قالَتَْ} (جَمَاعَة) \emphaseFF{الأَعرَاب}  \\ \exemplefont \underline{قَالَ} (جَمعُ) \emphaseFF{العُلَمَاء}  }}
            \LFTw{t}{\RL{العَامِل نِعمَ وَ بِئسَ \\ \exemplefont \underline{نِعمَتْ} \emphaseFF{المَرْأَةُ} هِندٌ \\ \exemplefont \underline{نِعمَ} \emphaseFF{المَرْأَةُ} هِندٌ }}
            \LFTw{t}{\RL{المُؤَنَّث اسم ظَاهِر حَقِيقِي التَّأنِيث\\"مُنفَصِل عَنِ العَامِل بِغَيرِ "إِلَّا \\ \exemplefont \underline{حَضَرَت} القَاضي \emphaseFF{امرَأَةٌ}   \\ \exemplefont \underline{حَضَرَ} القَاضي \emphaseFF{امرَأَةٌ}   }}
            \LFTw{t}{\RL{المُؤَنَّث اسم ظَاهِر مَجَازِي التَّأنِيث \\ \exemplefont \underline{طَلَعَت} \emphaseFF{الشَّمس}  \\ \exemplefont \underline{طَلَعَ} \emphaseFF{الشَّمس}  }}
          }}
      }}

  \end{center}
\end{preview}
\end{document}