% Generated by LaTreeX, <https:\\lautgesetz.com/latreex/>
% !TEX root = ../000_Complet/A4_004.tex
\documentclass{article}
\usepackage[a4paper,margin=1in,landscape]{geometry}
\usepackage[active,tightpage,xetex]{preview}
\usepackage[tuenc]{fontspec}
%\usepackage{pstricks,pst-node,pst-tree}
\usepackage{varwidth,realscripts}
\usepackage{graphicx}
\usepackage{bidi}
\usepackage{titling}
\defaultfontfeatures{Mapping=tex-text}
\setmainfont{Amiri-Bold}[Script=Arabic]
\pagestyle{empty}
%\psset{showbbox=false,
%       treemode=D,
%       linewidth=0.3pt,
%       treesep=2em,
%       levelsep=20ex,
%       arrows=->,
%       edge=\ncangles,
%       angleA=-90,
%       angleB=90,
%       armA=3ex,
%       armB=1ex}
%\newcommand{\LFTw}[2]{%
%  \TR[ref=#1]{\psframebox[linestyle=none,framesep=3pt]{%
%    \begin{varwidth}{15ex}\center #2\end{varwidth}}}
%}
%\newcommand{\LFTr}[2]{%
%  \TR[ref=#1]{%
%    \psframebox[linestyle=none,framesep=2pt]{#2}}
%}
\newcommand{\exemple}[1]{%
  \fontspec{Amiri}[Scale=0.9,Script=Arabic]
  \RL{#1}
}
\newfontfamily{\frfont}{Times New Roman}[Scale=0.7]
\newcommand{\LLR}[1]{%
  \LR{{\frfont #1}}
}
\newfontfamily{\exemplefont}{Amiri}[Scale=0.9,Script=Arabic]
\newcommand{\annot}[1]{%
  \fontspec{Amiri}[Scale=0.7,Script=Arabic]
  \RL{#1}
}

%\def\pstreehooki{\psset{thislevelsep=*0pt}}
%\def\pstreehookiii{\psset{thislevelsep=*0pt}}
%\def\pstreehookv{\psset{thislevelsep=*0pt}}
%\def\pstreehookvii{\psset{thislevelsep=*0pt}}
%\def\pstreehookix{\psset{thislevelsep=*0pt}}
%\def\pstreehookxi{\psset{thislevelsep=*0pt}}
%\def\pstreehookxiii{\psset{thislevelsep=*0pt}}
%\def\pstreehookxv{\psset{thislevelsep=*0pt}}

\pretitle{}\posttitle{}\title{}
\preauthor{}\postauthor{} 
\author{Cours 4 - \RL{صَرف  -  معاني الأفعال }}
\predate{}\postdate{}\date{}

\begin{document}
    \begin{preview}
        \begin{center}
            
        \end{center}
            \LARGE %small,normalsize,large,Large,LARGE,huge,HUGE
%pstree
%{\exemplefont \LLR{ - TextFR - } ان\\}
            \RL{

                الأَسماء الستة\\ 
                
                أبُوكَ / أَخُوكَ / حَمُوكِ / فُوكَ / ذُو مَالٍ / هَنُوكَ\\
                وَ لاَ تُعرَبُ بِالحُرُوف إِلاَّ بشُرُوط أَربَعَة :\\
                ١ أَن تَكُون مُفرَدة\\
                ٢ أَن تَكُون مُكبَّرة\\
                ٣ أَن تَكُون مُضَافة\\
                ٤ أَن تَكُون المُضَاف لِغَِيرِ(ياء) المُتَكَلِّم \\
                
                (ذُو)\\
                (فُوكَ)\\
                (هَنُوكَ) فَجُمهُور العَرَب يقُولُونَ (هَذَا هَنُكَ) وَ بَعدُهُوم (هَذَا هَنُوكَ)\\
                ٤ أَن تَكُون\\

}
    \end{preview}
  \end{document}